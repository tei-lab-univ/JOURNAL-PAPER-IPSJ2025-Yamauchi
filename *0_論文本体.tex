\documentclass[submit]{ipsj}
%\documentclass{ipsj}

\usepackage{graphicx}
\usepackage{latexsym}

%% add package
\usepackage{comment} %コメントアウト
\usepackage{algorithm} %アルゴリズム
\usepackage{algorithmic} %アルゴリズム
\usepackage{enumerate} %アイテマイズ
\usepackage{amssymb, amsmath} %数式の揃え
\usepackage{amsthm} %証明環境
\usepackage{booktabs} %表の線
\usepackage{multirow}
% \usepackage{ulem} %斜線
\usepackage[legacycolonsymbols]{mathtools}

\def\Underline{\setbox0\hbox\bgroup\let\\\endUnderline}
\def\endUnderline{\vphantom{y}\egroup\smash{\underline{\box0}}\\}
\def\|{\verb|}
\def\newblock{\hskip .11em plus .33em minus .07em}

\renewcommand{\proofname}{\textbf{証明}}

\setcounter{巻数}{59}
\setcounter{号数}{1}
\setcounter{page}{1}


\受付{2016}{3}{4}
% \再受付{2015}{7}{16}   %省略可能
% \再再受付{2015}{7}{20} %省略可能
% \再再受付{2015}{11}{20} %省略可能
\採録{2016}{8}{1}


% 新しい記号の作成
\makeatletter
% \newcommand{\process}{\mathrel{\vphantom{\vee}\mathpalette\process@\relax}}
% \newcommand{\process@}[2]{%
%   \wedge
% }
\newcommand{\sequence}{\mathrel{\vphantom{\wedge}\mathpalette\sequence@\relax}}
\newcommand{\sequence@}[2]{%
  \ooalign{\hidewidth$\m@th#1{\wedge}$\hidewidth\cr$\m@th#1{\vee}$\cr}%
}
\makeatother


\begin{document}

\title{ゲーム空間の一括構築を用いた段階的な部分合成による離散制御器合成の計算空間削減}

\etitle{Computational Space Reduction of Discrete Controller Synthesis by Stepwise Partial Synthesis with Consolidated Game Composition}

\affiliate{WASEDA}{早稲田大学 Waseda University}
\affiliate{TIT}{東京工業大学 Tokyo Institute of Technology}
\affiliate{NII}{国立情報学研究所 National Institute of Informatics (NII)}

\author{山内 拓人}{Takuto Yamauchi}{WASEDA,TIT}[takuto.yamauchi@aoni.waaseda.jp]
\author{李 家隆}{Jialong Li}{WASEDA}[]
\author{鄭 顕志}{Kenji Tei}{WASEDA,TIT,NII}[]
\author{本位田 真一}{Shinichi Honiden}{NII}[]

\begin{abstract}
想定される動作環境下で安全性が保証された動作仕様を自動合成するDiscrete Controller Synthesis(DCS)において,計算空間の指数爆発の課題はその実践的な適用を阻む主要因となっている.
本課題に対処するため,本論文では先行の計算空間削減手法を相補的に組み合わせたConsolidated Stepwise Discrete Controller Synthesis(CSDCS)を提案する.段階的なゲーム空間構築アプローチに違反状態空間の抽象化処理を採用することで,計算空間削減効果を維持したまま計算空間の構築回数を抑えることができる.
結果,従来トレードオフの関係にあった計算空間削減と計算時間削減の両方を高い水準で両立を可能にした.
その性能を形式的な妥当性証明と7つのシナリオを通して確認した結果,計算時間を平均43.6\%,計算空間を平均51.8\%削減しつつ従来のDCSと同じ機能を果たすことが可能であることがわかった.
\end{abstract}
\begin{jkeyword}
離散制御器合成,二人型対戦ゲーム理論,ラベル付き遷移システム
\end{jkeyword}


\begin{eabstract}
% 再度翻訳
% In Discrete Controller Synthesis, a technique for automatically generating operational specifications that ensure safety under assumed environments, state-space explosion remains a major obstacle to practical application.
% To address this issue, this paper proposes Consolidated Stepwise Discrete Controller Synthesis (CSDCS), which combines existing state-space reduction techniques in a complementary manner.
% CSDCS incorporates violation-state abstraction into the stepwise game-space construction approach proposed in prior work, thereby reducing the number of game constructions while maintaining the space reduction effect. This enables a balance between state-space reduction and computational efficiency.
% To evaluate CSDCS, we formally prove its correctness and conduct experiments based on seven scenarios. The results confirm that CSDCS achieves an average of 43.6\% reduction in computation time while maintaining a high state-space reduction effect of 51.8\% on average.

In Discrete Controller Synthesis (DCS), which automatically synthesizes behavior specifications that guarantee safety under assumed environments, the issue of state space explosion remains a major obstacle to its practical application.To address this issue, this paper proposes Consolidated Stepwise Discrete Controller Synthesis (CSDCS), which complements and integrates prior state space reduction techniques. By incorporating violation-state abstraction into a stepwise game space construction approach, the number of game space constructions can be reduced while maintaining the effectiveness of state space reduction. As a result, CSDCS achieves a high level of both state space reduction and computation time reduction, which were previously in a trade-off relationship. Formal validation and evaluations on seven scenarios demonstrated that CSDCS can achieve the same functionality as conventional DCS while reducing computation time by an average of 43.6\% and state space by an average of 51.8\%.
\end{eabstract}
\begin{ekeyword}
Discrete Controller Synthesis, Two-players Game Theory, Labeled Transition System
\end{ekeyword}

\maketitle

\theoremstyle{definition}
\newtheorem{highlight}{}
\newtheorem{thm}{定理}
\newtheorem{dfn}[thm]{定義}
\newtheorem{pro}{命題}
\newtheorem{lem}{補題}
\newtheorem{thx}{謝辞}


% 本文ここから
\section{まえがき}
\label{section:introduction}
事象の発生によりシステムの状態が離散的に遷移する離散事象システムの開発において,開発の早期段階において安全性が保証された動作仕様を定めることが重要となる.
想定される動作環境下において安全性が保証された動作仕様を自動合成する技術のひとつとして,Discrete Controller Synthesis(DCS)\cite{paper:SupervisoryControl}に関する研究が進められてきた.

DCSを用いた動作仕様設計は次の手順で行われる.はじめに,開発者はシステムの動作環境を仮定し,その特性を環境モデルとしてモデル化する.次に,環境モデル下で保証すべき安全性を定義し,その安全性の充足状況を監視する監視モデルを作成する.最後に,環境モデルと監視モデルを入力としてDCSを実施し,環境下で安全性が保証された動作仕様を表す制御器を自動合成する.DCSは,モデル検査やソフトウェアテストにおいて手動で行われていた動作仕様策定と検証を自動化することで,開発者の負担を軽減する.

しかしながら,DCSには計算空間が指数増加する課題があり,実践的な規模のシステム開発への適用を困難にしている.
DCSは,制御器を合成する過程で環境モデルと監視モデルから安全性を満たす状態空間を分析するためのゲーム空間を構築する.
このゲーム空間は,事象の同期を考慮した環境モデルと監視モデルの全状態の直積により状態を構築するため,それぞれのモデル数の増加に伴って状態数が指数的に増加する.
このゲーム空間の指数増加に伴って,要求される計算空間,計算時間,必要主記憶量も指数増加するため,DCSにおいて計算空間の状態削減は重要な課題となっている.

Consolidated Discrete Controller Synthesis(CDCS)\cite{yamauchi:IPSJ2024}とStepwise Discrete Controller Synthesis(SDCS)\cite{yamauchi:IEICEJ2023}はDCSのゲーム空間構築における状態空間の削減に取り組み,DCSにおける計算空間の指数爆発を抑制した.
CDCS\cite{yamauchi:IPSJ2024}では,監視モデルの違反状態が重ね合わせられた状態をひとつに抽象化しつつゲーム空間を構築するゲーム空間構築手法であるConsolidated Game Compositionを提案した.
ゲーム空間構築過程にあたって,安全性ゲームを解くにあたって探索することがない違反状態が重ね合わせられた複数の状態をひとつの状態抽象化することで計算空間の状態削減を実現した.
SDCS\cite{yamauchi:IEICEJ2023}では,監視モデルごとに必要最小限の構成でゲーム空間の構築と分析を行う部分合成を提案した.
そして,部分合成によって違反状態であると判明した状態を,他の監視モデルの安全性ゲームにおいて構築回避しつつ,段階的にゲーム空間を構築することで計算空間の状態削減を実現した.

これらCDCSとSDCSは,どちらも違反状態に着目したDCS計算空間削減手法であるが,どちらの手法が有効であるかは扱う環境モデルと監視モデルの組み合わせごとに異なる.
そのため,開発者は対象システムに応じてCDCSとSDCSを適切に使い分ける必要があるが,判断するには状態爆発を引き起こす中間生成モデルの構造をすべて把握する必要があり,どちらが有効であるか判断することは現実的に困難である.

そこで本論文では,CDCSとSDCSにおける適用選択の課題を解消するために,SDCSとCDCSを一元化したDCS,Consolidated Stepwise Discrete Controller Synthesis(CSDCS)を提案する.
CSDCSでは,SDCSおよびCDCSのいずれと比較しても同等以上の適用可能範囲と状態空間削減効果を備えるために,SDCSの段階的なゲーム空間構築アプローチを基盤としつつ,各ステップにおけるゲーム空間構築処理にConsolidated Game Compositionを採用する.
Consolidated Game Compositionの採用によって,複数の監視モデルを同時に分析したとしてもゲーム空間の段階構築による計算空間削減効果の減少を抑制できるようになる.よってCSDCSでは,高い計算空間削減効果を維持しつつ,部分合成回数を減らすことが可能となったため,SDCSと比較して少ない計算時間で同等以上の計算空間削減効果を実現した.
このCSDCSによって,従来必要であったSDCSとCDCSの使い分けに関する判断工程が不要とし,両者の適用選択における課題を解消する.

本論文の貢献は以下の通りである.
\begin{itemize}
	\item CSDCSの合成アルゴリズムの定式化
	\item CSDCSが合成する制御器の妥当性証明
	\item CSDCSの計算空間削減効果の性能評価
\end{itemize}
CSDCSを評価するにあたって,評価実験としてSDCSやCDCSの評価で用いられた7つの離散事象システムをテストベッドとし,SDCS,CDCS,CSDCSで制御器を合成した.合成で要求される計算空間,必要主記憶量,計算時間を計測し,SDCS,CDCS,CSDCSの性能を比較した.実験の結果,CSDCSは,SDCSとCDCSの両方の削減可能範囲を包括し,SDCSとCDCSの両方と同程度以上安定して高い削減効果量を持つことが確認されたため,SDCSとCDCSの適用選択の課題を解消しうることがわかった.

本論文の構成は以下の通りである.
\ref{section:DCS}章で背景技術であるDCSについて詳説する.
続く\ref{section:advDCS}章ではSDCSとCDCSがどのようにDCSの計算空間を削減したかについて詳説し,SDCSとCDCSにおける適用選択の課題について示す.
その後,適用選択の課題を解消するCSDCSの仕組みについて\ref{section:proposal}章で詳説し,CSDCSがSDCSとCDCSにおける適用選択の課題を解消しうるかについて\ref{section:evaluation}章で評価実験を通して評価する.
最後に,\ref{section:relatedwork}章で関連研究の説明の後,\ref{section:conclusion}章で結論を述べる.
\section{Discrete Controller Synthesis(DCS)}
\label{section:DCS}
本章では,与えられた環境下で安全性を保証する制御器を自動合成する技術であるDCSについて詳説する.

\subsection{DCSにおける事前準備}
\label{subsection:preparation}
DCSを用いた動作仕様設計は次の手順で行われる.はじめに,開発者はシステムの動作環境を仮定し,その特性を環境モデル$e$としてモデル化する.
次に,環境モデル$e$下で保証すべき安全性を定義し,その安全性の充足状況を監視する監視モデル$r$を作成する.
そして,環境モデル$e$と監視モデル$r$を入力としてDCSを実施することで,環境下で安全性が保証された動作仕様を表す制御器$c$を自動合成する.
そのようなDCSにおける$e$,$r$,$c$は次に定義されるLTSを用いて表現される.

\begin{dfn}{\textbf{LTS}}
\label{def:component_model}
    LTSは$x = (S_{x}, A_{x}, \Delta_{x}, s^0_{x})$で表現される.
    $S_{x}$は有限の状態集合であり,$s^0_{x}$は初期状態である.
    $A_{x} = A^+ \cup A^-$は事象の集合であり,$A^+$はシステムが制御可能な事象,$A^-$はシステムが制御不可能な事象である.
    $(s,a,s') \in \Delta_{x}$は遷移であり,事象$a \in A_{x}$によって状態$s \in S_{x}$は状態$s' \in S_{x}$へ遷移することを意味する.
    LTSの集合は大文字で$X$と表され,LTSの要素に下付き$X$と表記した$S_{X}$,$A_{X}$,$\Delta_{X}$,$s^0_{X}$は,$X$に含まれる全LTS$x \in X$の要素($S_{x}$,$A_{x}$,$\Delta_{x}$,$s^0_{x}$)の和集合を表す.
\end{dfn}

本論文では,LTSを区別するために,扱う全てのLTSにユニークな記号$id$を付与する.
よって,本論文で扱う環境モデルは決定性LTS $e = (id_{e}, S_{e}, A_{e}, \Delta_{e}, s^0_{e})$,監視モデルは決定性LTS $r = (id_{r}, S_{r}, A_{r}, \Delta_{r}, s^0_{r})$で表され,それぞれ非決定性を持たない.また,環境モデルの集合は$e \in E$,監視モデルの集合は$r \in R$で表される.
このとき,DCSで与えられる$E$と$R$は常に次の関係式全てを満たす.
\begin{enumerate}[\bf 関係式A]
\item $s_{err} \in S_{r}$
\item $A_{R} \subset A_{E}$
\item $\forall x, x' \in E \cup R:  x \neq x'  \Rightarrow S_x \cap S_{x'} = \emptyset$
\end{enumerate}
関係式Aは,$r$の状態$S_{r}$には安全性を違反する状態を表す違反状態$s_{err}$が含まれることを表す.
関係式Bは,$R$は$E$を踏まえて設計されるため,$a \in A_{R} \backslash A_{E}$となる事象$a$は存在しないことを表す.
関係式Cは,$E$と$R$に含まれるすべてのLTSの状態は共通要素を持たないことを表す.

\subsection{basic DCS}
\label{subsection:DCS algorithm}
DCSは,関係式A,B,Cを満たす$E$と$R$を入力として,安全性が保証されたシステムの制御器$c$を自動合成する.
$c$は,決定性LTS $c = (id_{c}, S_{c}, A_{c}, \Delta_{c}, s^0_{c})$で表され,
システムを構成する全ての$e \in E$において,どのように制御不可能な事象が生じたとしても,いずれの$r \in R$の$s_{err}$に遷移することのない,システムの制御を表したLTSである.
この性質から,$c$は$r$で表現された安全性を違反する状態に到達することのないシステムの動作仕様を表す.
この$c$の導出する基本的なDCS(以降,basic DCSと呼ぶ)\cite{paper:SynthesisOfLiveBehaviourModels}\cite{paper:SynthesisOfRun-To-CompletionControllers}\cite{paper:Concurrency}はAlgorithm\ref{algorithm:basic_controller_synthesis}で表される.

\begin{algorithm}[h]
\caption{basic DCS}
\label{algorithm:basic_controller_synthesis}
\begin{algorithmic}[1]
\renewcommand{\algorithmicrequire}{\textbf{Input:}}
\renewcommand{\algorithmicensure}{\textbf{Output:}}
\REQUIRE $E$ {\bf such that} $\forall e \in E, e = (id_{e}, S_{e}, A_{e}, \Delta_{e}, s^0_{e})$
\REQUIRE $R$ {\bf such that} $\forall r \in R, r = (id_{r}, S_{r}, A_{r}, \Delta_{r}, s^0_{r})$
\ENSURE  $c$

\STATE $m   \Leftarrow$ {\bf Parallel Composition($E$)}
\STATE $g   \Leftarrow$ {\bf Modified Parallel Composition($m$, $R$)}
\STATE $g^* \Leftarrow$ {\bf Error State Abstraction($g$)}
\STATE $c   \Leftarrow$ {\bf Safety Game Solving($g^*$)}
\STATE return $c$
\end{algorithmic}
\end{algorithm}

Algorithm\ref{algorithm:basic_controller_synthesis}では,はじめにParallel Compositionを用いて,$e \in E$からシステムの全環境がモデル化された決定性LTS $m = (S_{m}, A_{m}, \Delta_{m}, s^0_{m})$を構築する(1行目).
$m$を合成するParallel Compositionは定義\ref{def:parallel_composition}で定義される.
\begin{dfn}{\textbf{Parallel Composition}}
\label{def:parallel_composition}
    記号$\parallel$で表され,複数のLTSを同期されたひとつのLTSに統合する.
    入力のLTSを$S_{x} \cap S_{y} = \emptyset$を満たす$x = (S_{x}, A_{x}, \Delta_{x}, s^0_{x})$,
    $y = (S_{y}, A_{y}, \Delta_{y}, s^0_{y})$とする.
    $m = x \parallel y = (S, A, \Delta, s^0)$の時,初期状態$s^0$は状態の集合$\{ s^0_{x}, s^0_{y} \}$で表され,遷移$(s,a,s')\in\Delta$は式\ref{formula:pc1},\ref{formula:pc2},\ref{formula:pc3}を満たす初期状態$s^0$から到達可能な全状態間を結ぶ全遷移の集合である.
    \begin{multline}
    \label{formula:pc1}
    \Delta_x \langle s_{x} \overset{a}{\rightarrow} s'_{x} \rangle, \Delta_y \langle s_{y} \overset{a}{\nrightarrow} s'_{y} \rangle, a \in A_{x} \backslash A_{y}\\
    \Rightarrow (\{ s_{x},s_{y} \},a,\{ s'_{x},s_{y} \} ) \in \Delta
    \end{multline}
    \begin{multline}
    \label{formula:pc2}
    \Delta_x \langle s_{x} \overset{a}{\nrightarrow} s'_{x} \rangle, \Delta_y \langle s_{y} \overset{a}{\rightarrow} s'_{y} \rangle, a \in A_{y} \backslash A_{x}\\
    \Rightarrow (\{ s_{x},s_{y} \},a,\{ s_{x},s'_{y} \}) \in \Delta
    \end{multline}
    \begin{multline}
    \label{formula:pc3}
    \Delta_x \langle s_{x} \overset{a}{\rightarrow} s'_{x} \rangle, \Delta_y \langle s_{y} \overset{a}{\rightarrow} s'_{y} \rangle, a \in A_{x} \cap A_{y}\\
    \Rightarrow (\{ s_{x},s_{y} \},a,\{ s'_{x},s'_{y} \}) \in \Delta
    \end{multline}
    $\Delta_x \langle s_{x} \overset{a}{\rightarrow} s'_{x} \rangle$は$s_{x}$において$a$が生じた場合,状態が$s'_{x}$となる遷移が$\Delta_x$に含まれることを表し,$\Delta_x \langle s_{x} \overset{a}{\nrightarrow} s'_{x} \rangle$は$s_{x}$において$a$が生じた場合,状態が$s'_{x}$となる遷移が$\Delta_x$に含まれないことを表す.
    $S$は$\Delta$に含まれる全状態の集合,$A$は$\Delta$に含まれる全事象の集合で表され,$\Delta$が合成されたとき一意に定まる.
    $E = \{e_1, e_2, \ldots, e_n\}$の時,"{\bf Parallel Composition($E$)}"は$e_1 \parallel e_2 \parallel \cdots \parallel e_n$を表す.
\end{dfn}

次に,Modified Parallel Compositionを用いて,決定性LTSであるゲーム空間 $g = (S_{g}, A_{g}, \Delta_{g}, s^0_{g})$を構築する(2行目).
$g$は$m$において保証したい安全性$r$を違反する状態を重ね合わせたLTSであり,$g$において二人型対戦ゲーム理論に基づいた安全性ゲームを解くことで制御器 $c$を導出できる.この$g$を合成するModified Parallel Compositionは定義\ref{def:modified_parallel_composition}で定義される.
\begin{dfn}{\textbf{Modified Parallel Composition}}
\label{def:modified_parallel_composition}
    プロセス記号$\parallel_*$で表され,監視モデル$R$とその監視対象である監視対象モデル$m$からゲーム空間$g$を合成する.
    入力を$S_{r} \cap S_{m} = \emptyset$を満たす$r = (S_{r}, A_{r}, \Delta_{r}, s^0_{r}) \in R$と,
    $m = (S_{m}, A_{m}, \Delta_{m}, s^0_{m})$とする.
    $g = m \parallel_* r = (S, A, \Delta, s^0)$の時,$s^0$は$\{ s^0_{m}, s^0_{r} \}$で表され,遷移$(s,a,s')\in\Delta$は式\ref{formula:mpc1},\ref{formula:mpc2}を満たす初期状態$s^0$から到達可能な全状態間を結ぶ全遷移の集合である.
    \begin{multline}
    \label{formula:mpc1}
    \Delta_m \langle s_{m} \overset{a}{\rightarrow} s'_{m} \rangle, \Delta_r \langle s_{r} \overset{a}{\nrightarrow} s'_{r} \rangle, a \in A_{m} \backslash A_{r}\\
    \Rightarrow (\{ s_{m},s_{r} \},a,\{ s'_{m},s_{r} \} ) \in \Delta
    \end{multline}
    \begin{multline}
    \label{formula:mpc2}
    \Delta_m \langle s_{m} \overset{a}{\rightarrow} s'_{m} \rangle, \Delta_r \langle s_{r} \overset{a}{\rightarrow} s'_{r} \rangle, a \in A_{m} \cap A_{r}\\
    \Rightarrow (\{ s_{m},s_{r} \},a,\{ s'_{m},s'_{r} \}) \in \Delta
    \end{multline}
    % $\Delta_r \langle s_{r} \overset{a}{\rightarrow} s'_{r} \rangle$は$s_{r}$において$a$が生じた場合,状態が$s'_{r}$となる遷移が$\Delta_r$において含まれることを表し,$\Delta_r \langle s_{r} \overset{a}{\nrightarrow} s'_{r} \rangle$は$s_{r}$において$a$が生じた場合,状態が$s'_{r}$となる遷移が$\Delta_x$において含まれないことを表す.
    $S$は$\Delta$に含まれる全状態の集合,$A$は$\Delta$に含まれる全事象の集合で表され,$\Delta$が合成されたとき一意に定まる.
    また,$m = \{S_{m}, A_{m}, \Delta_{m}, s^0_{m}\}$,$R = \{r_1, r_2, \ldots, r_n\}$の時,"{\bf Modified Parallel Composition($m$, $R$)}"は$m \parallel_* r_1 \parallel_* r_2 \parallel_* \cdots \parallel_* r_n$を表す.
\end{dfn}

その後,Error State Abstractionを用いて,$g$から決定性LTS,$g^* = (S_{g^*}, A_{g^*}, \Delta_{g^*}, s^0_{g^*})$を構築する(3行目).$g^*$は$s_{err}$が重ね合わされた$g$の全状態をひとつの$s_{err}$に圧縮したLTSであり,$g$を$g^*$に一度変換することでSafety Game Solvingにおける計算量が削減される.
そのError State Abstractionは定義\ref{def:error_state_abstraction}で定義される.
\begin{dfn}{\textbf{Error State Abstraction}}
\label{def:error_state_abstraction}
    ゲーム空間$g$において$s_{err}$を含む全ての状態をひとつの$s_{err}$に置換し,状態空間を圧縮する.
    入力のゲーム空間を$g = \{S_{g}, A_{g}, \Delta_{g}, s^0_{g}\}$,出力のゲーム空間を$g^* = \{S_{g^*}, A_{g^*}, \Delta_{g^*}, s^0_{g^*}\}$とする.
    $s^0_{g^*}$は$s^0_{g^*} = s^0_{g}$となり,
    $S_{g^*}$は$S_{g^*} = \{s_{err}\} \cup \{ s_{g} \in S_{g} \mid s \notin s_{err} \}$によって導出される.
    $\Delta_{g^*}$は$\Delta_{g^*} = \{ (s,a,s') \in \Delta_{g} \mid s \in S_{g^*} \}$で導出されるが,導出の過程で$\Delta_{g^*}$に含まれる$s_{err}$を含む全ての状態は$s_{err}$に置換される.
    また,$A_{g}$は$A_{g} = { a \mid (s,a,s') \in \Delta_{g}}$と表される.
    以上の$g$から$g^*$を導出する作業を"{\bf State Abstruction($g$)}"で表す.
\end{dfn}

最後に,$g^*$においてSafety Game Solvingすることで制御機$c$を合成する(4行目).本論文における「分析」はこのSafety Game Solvingを指し,$g^*$において,どのように制御不可能な事象が生じたとしても$s_{err}$に到達することのない全状態遷移を導出し$c$とする.
Safety Game Solvingは定義\ref{def:safety_game_solving}で定義される.

\begin{dfn}{\textbf{Safety Game Solving}}
\label{def:safety_game_solving}
    ゲーム空間$g^* = (S_{g^*}, A_{g^*}, \Delta_{g^*}, s^0_{g^*})$を入力とし,制御器$c$を出力する.
    二人型対戦ゲーム理論を用いて$s_{err} \in S_{g^*}$から逆伝搬的に$g^*$の状態の探索を行い,
    $A^+$をどのように発生させたとしても$A^-$次第で$s_{err}$に遷移しうる状態である全ての状態を見つけ,$S_{err}$に分類する.
    その後,$S_{g^*}$のうち,$S_{err}$に含まれない全状態を状態の集合$S_{safe}$に分類し,$\Delta_{g}$のうち$S_{safe}$に含まれる状態のみで構築された遷移$\Delta_{safe}$を特定する.
    $\Delta_{safe}$を構成する全ての事象を$A_{safe}$とする.
    特定した$S_{safe}$,$\Delta_{safe}$,$A_{safe}$から制御器$c=\{S_{safe}, A_{c}, \Delta_{safe}, s^0_{g^*}\}$を構築し出力する.
    以上の$g^*$から$c$を導出する作業を"{\bf Safety Game Solving($g^*$)}"で表す.
\end{dfn}

以上のParallel Composition,Modified Parallel Composition,Error State Abstraction,Safety Game Solvingを通して$c$は合成される.
このとき,$c$の導出過程で構築される$m$,$g$,$g^*$の状態数は入力される環境モデルの数に応じて指数増加する.
その結果,DCSに必要となる主記憶量や計算時間も指数的に増大するため,DCSを実践規模のシステムへ適用していく上で,この計算空間の指数増加を抑制することは重要な課題だとされてきた\cite{paper:DirectedControllerSyntehsis}\cite{aizawa:IEICEJ2020}\cite{aizawa:SmartWorld}.
この課題に対し,SDCS\cite{yamauchi:IEICEJ2023}とCDCS\cite{yamauchi:IPSJ2024}は,$c$合成過程の冗長な計算空間を特定し,構築を回避するDCSアルゴリズムを提案することで,計算空間の指数爆発の抑制に取り組んだ.


\section{計算空間の指数爆発抑制に取り組んだDCS}
\label{section:advDCS}
本章では,basicDCSにおける計算空間の指数爆発の抑制に取り組んだDCS,CDCSとSDCSについて詳説する.

% ---------------------------------------------------------------------------------------------------------- %

\subsection{CDCS}
\label{subsection:CDCS}
CDCS\cite{yamauchi:IPSJ2024}は,監視モデルの$s_{err}$に該当する全ての状態を,ひとつの$s_{err}$に抽象化しつつ,ゲーム空間構築することで,DCSの計算空間を削減する.
安全性ゲームを解く上で,ゲーム空間の$s_{err}$から伸びる遷移は探索する必要がない.
そこで,ゲーム空間構築過程で複雑な$s_{err}$の構築を回避することで,計算空間削減を実現する.
CDCSのアルゴリズムをAlgorithm\ref{algorithm:CDCS}に示す.

\begin{algorithm}[h]
\caption{CDCS}
\label{algorithm:CDCS}
\begin{algorithmic}[1]
\renewcommand{\algorithmicrequire}{\textbf{Input:}}
\renewcommand{\algorithmicensure}{\textbf{Output:}}
\REQUIRE $E$ {\bf such that} $\forall e \in E, e = (id_{e}, S_{e}, A_{e}, \Delta_{e}, s^0_{e})$
\REQUIRE $R$ {\bf such that} $\forall r \in R, r = (id_{r}, S_{r}, A_{r}, \Delta_{r}, s^0_{r})$
\ENSURE  $c$

\STATE $g^* =$ {\bf Consolidated Game Composition($E \cup R$})
\STATE $c   =$ {\bf Safety Game Solving($g^*$)}
\STATE return $c$
\end{algorithmic}
\end{algorithm}

Algorithm\ref{algorithm:CDCS}では,Algorithm\ref{algorithm:basic_controller_synthesis}におけるParallel Composition,Modified Parallel Composition,Error State Abstraction全てを一括で行うConsolidated Game Compositionを用いて,$E$と$R$から直接$g^*$を構築する(1行目).その後,basic DCSと同様に$g^*$を入力としてSafety Game Solvingを用いて$c$を導出する.
CDCSを実現するConsolidated Game Compositionは定義\ref{def:consolidated_game_composition}で定義される.

\begin{dfn}{\textbf{Consolidated Game Composition}}
\label{def:consolidated_game_composition}
    記号$\between$で表され,環境モデル群$E$と監視モデル群$R$に含まれる全てのLTSの集合 $E \cup R$を入力とし,$g^*$を直接構築し,出力する.
    LTS $x,y \in E \cup R$が入力される時,$x = (S_{x}, A_{x}, \Delta_{x}, s^0_{x})$,
    $y = (S_{y}, A_{y}, \Delta_{y}, s^0_{y})$とする.
    $g^* = x \between y = (S, A, \Delta, s^0)$の時,初期状態$s^0$は状態の集合$\{ s^0_{x}, s^0_{y} \}$で表され,遷移$(s,a,s')\in\Delta$は式\ref{formula:pc1},\ref{formula:pc2},\ref{formula:pc3}を満たす初期状態$s^0$から到達可能な全状態間を結ぶ全遷移の集合である.
    \begin{multline}
    \label{formula:pc1}
    \Delta_x \langle s_{x} \overset{a}{\rightarrow} s'_{x} \rangle, \Delta_y \langle s_{y} \overset{a}{\nrightarrow} s'_{y} \rangle, a \in A_{x} \backslash A_{y}, s_{err} \notin s'_{x}\\
    \Rightarrow (\{ s_{x},s_{y} \},a,\{ s'_{x},s_{y} \} ) \in \Delta
    \end{multline}
    \begin{multline}
    \label{formula:pc2}
    \Delta_x \langle s_{x} \overset{a}{\nrightarrow} s'_{x} \rangle, \Delta_y \langle s_{y} \overset{a}{\rightarrow} s'_{y} \rangle, a \in A_{y} \backslash A_{x}, s_{err} \notin s'_{y}\\
    \Rightarrow (\{ s_{x},s_{y} \},a,\{ s_{x},s'_{y} \}) \in \Delta
    \end{multline}
    \begin{multline}
    \label{formula:pc3}
    \Delta_x \langle s_{x} \overset{a}{\rightarrow} s'_{x} \rangle, \Delta_y \langle s_{y} \overset{a}{\rightarrow} s'_{y} \rangle, a \in A_{x} \cap A_{y}, s_{err} \notin s'_{x} \cup s'_{y}\\
    \Rightarrow (\{ s_{x},s_{y} \},a,\{ s'_{x},s'_{y} \}) \in \Delta
    \end{multline}
    \begin{multline}
    \label{formula:pc4}
    \Delta_x \langle s_{x} \overset{a}{\rightarrow} s'_{x} \rangle, \Delta_y \langle s_{y} \overset{a}{\nrightarrow} s'_{y} \rangle, a \in A_{x} \backslash A_{y}, s_{err} \in s'_{x}\\
    \Rightarrow (\{ s_{x},s_{y} \},a,s_{err}) \in \Delta
    \end{multline}
    \begin{multline}
    \label{formula:pc5}
    \Delta_x \langle s_{x} \overset{a}{\nrightarrow} s'_{x} \rangle, \Delta_y \langle s_{y} \overset{a}{\rightarrow} s'_{y} \rangle, a \in A_{y} \backslash A_{x}, s_{err} \in s'_{y}\\
    \Rightarrow (\{ s_{x},s_{y} \},a,s_{err}) \in \Delta
    \end{multline}
    \begin{multline}
    \label{formula:pc6}
    \Delta_x \langle s_{x} \overset{a}{\rightarrow} s'_{x} \rangle, \Delta_y \langle s_{y} \overset{a}{\rightarrow} s'_{y} \rangle, a \in A_{x} \cap A_{y}, s_{err} \in s'_{x} \cup s'_{y}\\
    \Rightarrow (\{ s_{x},s_{y} \},a,s_{err}) \in \Delta
    \end{multline}
    % $\Delta_x \langle s_{x} \overset{a}{\rightarrow} s'_{x} \rangle$は$s_{x}$において$a$が生じた場合,状態が$s'_{x}$となる遷移が$\Delta_x$に含まれることを表し,$\Delta_x \langle s_{x} \overset{a}{\nrightarrow} s'_{x} \rangle$は$s_{x}$において$a$が生じた場合,状態が$s'_{x}$となる遷移が$\Delta_x$に含まれないことを表す.
    $S$は$\Delta$に含まれる全状態の集合,$A$は$\Delta$に含まれる全事象の集合で表され,$\Delta$が合成されたとき一意に定まる.
    以上の$E$と$R$から$g^*$を直接導出する作業を"{\bf Consolidated Game Composition ($E \cup R$)}"で表す.
\end{dfn}

このConsolidated Game Compositionを用いることによって,$s_{err}$が重ね合わされる全状態をひとつの$s_{err}$として圧縮しつつ,$g^*$を構築できる.これにより,$s_{err}$が重ね合わされる状態の数だけ計算空間が削減される.


% ---------------------------------------------------------------------------------------------------------- %

\subsection{SDCS}
\label{subsection:SDCS}
SDCS\cite{yamauchi:IEICEJ2023}は,部分合成を段階的に行うことで,DCSの計算空間を削減する.
部分合成とは,システムの部分的な制御器を合成する工程を意味し,システムを構成する一部の環境モデルと監視モデルにParallel Composition,Modified Parallel Composition,Error State Abstraction,Safety Game Solvingを適用することで実現される.
部分制御器を用いてゲーム空間の構築と分析を段階的に行うことで,部分違反状態(先の分析で安全性を違反すると判明した状態)の構築を以降全てのゲーム空間の構築・分析で回避でき,計算空間削減を実現する.
SDCSのアルゴリズムをAlgorithm\ref{algorithm:SDCS}に示す.

\begin{algorithm}[h]
\caption{SDCS}
\label{algorithm:SDCS}
\begin{algorithmic}[1]
\renewcommand{\algorithmicrequire}{\textbf{Input:}}
\renewcommand{\algorithmicensure}{\textbf{Output:}}
\REQUIRE $E$ {\bf such that} $\forall e \in E, e = (id_{e}, S_{e}, A_{e}, \Delta_{e}, s^0_{e})$
\REQUIRE $R$ {\bf such that} $\forall r \in R, r = (id_{r}, S_{r}, A_{r}, \Delta_{r}, s^0_{r})$
\ENSURE $c$
\STATE $\Lambda =$ {\bf Algorithm \ref{algorithm:huristic}($E$, $R$, $1$)}
\FOR {$i^* = 1$ to $n(\Lambda)$}
    \STATE $\lambda = (i^*, eids_{\lambda}, rids_{\lambda}, cid_{\lambda}),  \in \Lambda$
    \STATE $R^* = \{\forall r \in R \mid id_{r} \in rids_{\lambda}\}$
    \STATE $E^* = \{\forall e \in E \mid A_{e} \cap A_{R^*} \neq \emptyset\}$
    \STATE $m   =$ {\bf Parallel Composition($E^*$)}
    \STATE $g   =$ {\bf Modified Parallel Composition($m$, $R^*$)}
    \STATE $g^* =$ {\bf Error State Abstraction($g$)}
    \STATE $c   =$ {\bf Safety Game Solving($g^*$)}
    \STATE $c^* = (cid_{\lambda}, S_{c}, A_{c}, \Delta_{c}, s^0_{c})$
    \STATE $R   = R \backslash R^*$,\;\;\ $E = E \backslash E^* \cup \{c^*\}$
\ENDFOR
\STATE {\bf return} $c^* \in E$
\end{algorithmic}
\end{algorithm}

Algorithm\ref{algorithm:SDCS}では,$\Lambda$に則って,繰り返し部分合成(6-9行目)を適用することで$c$を導出する.
このとき,$\Lambda$は定義\ref{def:synthesis_sequence}で定義され,Algorithm\ref{algorithm:huristic}で導出される(1行目)\cite{yamauchi:KBSE2024}.

\begin{dfn}{\textbf{合成シーケンス }}
\label{def:synthesis_sequence}
    合成シーケンス$\Lambda$は合成プロセス$\lambda = (i_{\lambda}, eids_{\lambda}, rids_{\lambda}, cid_{\lambda})$の集合によって表される.
    $i_{\lambda}$は$\Lambda$において何番目に実行する$\lambda$であるかを表し,1から連続する整数が与えられる.
    $eids_{\lambda}$は$\lambda$で入力される環境モデル(LTS)の$id$の集合,$rids_{\lambda}$は$\lambda$で入力される監視モデル(LTS)の$id$の集合を表す.$cid_{\lambda}$は$\lambda$の出力となる部分制御器(LTS)$c$の$id$を表し,$\lambda$において$c$を合成した際には$id_{c}$にこの$cid_{\lambda}$を割り当てる.
\end{dfn}

\begin{algorithm}[h]
\caption{監視対象モデル数を考慮した$\Lambda$の合成}
\label{algorithm:huristic}
\begin{algorithmic}[1]
\renewcommand{\algorithmicrequire}{\textbf{Input:}}
\renewcommand{\algorithmicensure}{\textbf{Output:}}
\REQUIRE $E$ {\bf such that} $\forall e \in E, e = (id_{e}, S_{e}, A_{e}, \Delta_{e}, s^0_{e})$
\REQUIRE $R$ {\bf such that} $\forall r \in R, r = (id_{r}, S_{r}, A_{r}, \Delta_{r}, s^0_{r})$, $i^*$
\ENSURE  $\Lambda$
\STATE $\mu^*= \infty$,\;\; $cid^* = ${\bf uniqueID}
\FORALL{$r \in R$}
    \STATE $\mu = n(\{\forall e \in E \mid A_e \cap A_r \neq \emptyset\})$
    \IF{$\mu < \mu^*$}
        \STATE $r^* = r$,\;\; $\mu^* = \mu$
    \ENDIF
\ENDFOR
\STATE $E^* = \{\forall e \in E \mid A_{e} \cap A_{R^*} \neq \emptyset\}$
\STATE $\lambda^* = \{(i^*, \{\forall id_{e} \mid (id_{e}, S_{e}, A_{e}, \Delta_{e}, s^0_{e}) \in E^*\}, \{id_{r^*}\}, cid^*)\}$
\STATE $c^* = (cid^*, \emptyset, A_{E^*}, \emptyset, \emptyset)$
\IF{$n(R)=1$}
\STATE {\bf return} $\{\lambda^*\}$
\ENDIF
\STATE {\bf return} $\{\lambda^*\}$ $\cup$ {\bf Algorithm \ref{algorithm:huristic}($E \backslash E^* \cup \{c^*\}$, $R \backslash \{r^*\}$, $i^*+1$)}
\end{algorithmic}
\end{algorithm}

Algorithm\ref{algorithm:huristic}では,全監視モデルのうち$\mu$が最小となる監視モデル$r \in R$を$r^*$として導出し(2-7行目),全環境モデルのうち$r^*$の事象を含む全ての環境モデルを$E^*$として導出する(8行目).
そして,$r^*$と$E^*$から$i^*$ステップ目に部分合成する合成プロセスを構築する(9行目).
その後,$r^*$と$E^*$を部分合成した際に合成される制御器$c^*$を以降の合成プロセスの導出で必要な情報($id_{c^*}$,$A_{c^*}$)だけ埋めて仮に構築し(10行目),$i^*+1$ステップ目の合成プロセスの導出する(14行目).これを監視対象モデル数の数だけ繰り返し実施する(11-13行目).

以上のようにして導出された$\Lambda$に基づいて,Algorithm\ref{algorithm:huristic}では$i_{\lambda}$が1の$\lambda \in \Lambda$から順に繰り返し部分合成を行う(2-12行目).部分合成では,はじめに,入力された$R$のうち$rids_{\lambda}$に$id_{r}$が含まれる$r$の集合を$R^*$とし(4行目),$R^*$と$E$から部分合成の適用可能条件式$A_{r} \subseteq A_{E^*} \backslash A_{E \backslash E^*}$を満たす$e \in E$の集合を$E^*$として導出する(5行目).その後,導出された$R^*$と$E^*$を用いてParallel Composition(6行目),Modified Parallel Composition(7行目),Error State Abstraction(8行目),Safety Game Solving(9行目)を適用し$c$を合成する.そして,合成された$c$の$id$に$cid_{\lambda}$を割り当て$c^*$を用意する(10行目).最後に,次の$\lambda$で使用する$R$と$E$から使用した$R^*$と$E^*$を取り除き,$c^*$を$E$に加える(11行目).これにより,$i^*+1$ステップ目以降の部分合成において,$i^*$ステップ目の$R^*$を違反する状態(部分違反状態)
の構築を回避することができる.

% ---------------------------------------------------------------------------------------------------------- %

\subsection{CDCSとSDCSにおける適用選択の課題}
\label{subsection:limitation}
CDCSやSDCSを用いて,basic DCSよりも効率よく制御器を合成するにあたり,開発者は以下の制約を考慮しつつ適切にCDCSやSDCSを使い分ける必要がある.

\begin{enumerate}[\bf 制約1]
\item CDCSまたはSDCSのいずれか一方のみが有効となる場合がある
\item[\bf 制約2]  SDCSはbasic DCSよりも計算時間が増加(性能が悪化)する可能性がある
\end{enumerate}

制約1は,CDCSとSDCSがそれぞれ異なる方針で計算空間の削減を行うため,削減可能な状態が異なり,適用可能範囲が一致しないことに起因する.
制約2は、SDCSにおいて計算空間の削減効果が限定的な場合,段階的なゲーム空間$g^{*}$の構築・分析を繰り返すオーバヘッドが相対的に大きくなり,全体の計算時間が悪化することに起因する.

以上の制約を踏まえて,CDCSとSDCSの適用選択は慎重に判断する必要がある.
しかし,判断には,各手法により生成される中間生成LTS,$m$,$g$,$g^{*}$の構造を詳細に把握する必要があり,これを入力される環境モデルと監視モデルの組み合わせに応じて手作業で行うことは実運用上の大きな障壁となっている.
本論文では,このSDCSとCDCSの適用選択の課題を解消することを目的とする.

% しかし,現状でこれらの制約を考慮するには,各手法で生成される複雑な中間生成LTS($m$,$g$,$g^{*}$)をシステムごとに手作業で把握するほかなく,CDCSやSDCSのどちらを適用するか判断する上で障壁となっている.
% 制約2は,SDCSにおいて,計算空間の削減効果が小さい場合,削減される計算時間よりも繰り返し$g^{*}$を構築・分析するオーバヘッドが優位となることに起因する.
% 上記制約を考慮してCDCSとSDCSの適用選択を慎重に判断する必要があるが,上記制約を考慮するためには各手法で生成される複雑な中間生成LTS($m$,$g$,$g^{*}$)を手作業で把握する他なく,実運用上の障壁となっている.

% 制約1は,CDCSとSDCSで削減できる計算空間が異なる(適用可能範囲が異なる)ことに起因する.

% CDCSは監視モデルにおける違反状態をひとつの状態に抽象化しつつ$g^{*}$を構築するため以下の特徴がある.
% \begin{enumerate}[\bf 削減対象:]
% \item $m$,$g$($g^{*}$構築過程で生成されるLTS)
% \item[\bf 限界:] $g^*$の構築過程のみを改善するため,basic DCSと比較して$g^{*}$の削減効果は得られない
% \end{enumerate}

% 一方,SDCSは監視モデルごとに繰り返し$g^{*}$を構築・分析するため以下の特徴がある.
% \begin{enumerate}[\bf 削減対象:]
% \item $m$,$g$,$g^{*}$(全ての中間生成LTS)
% \item[\bf 限界:] 繰り返し$g^{*}$を構築・分析するため,計算空間の削減効果が小さい場合はbasic DCSよりも多くの計算時間を要する
% \end{enumerate}

% 以上のように,SDCSとCDCSでは削減対象となる中間生成LTS($m$,$g$,$g^{*}$)が異なる.
% この削減対象の違いは,それぞれの手法が採用する削減アプローチの違いによるものであるが,どちらがより有効に機能するかは,扱う環境モデルおよび監視モデルの組み合わせに依存する.
% 特に,SDCSは繰り返し$g^{*}$を構築・分析する性質上,計算空間の削減効果が小さい場合には,かえってbasic DCSよりも多くの計算時間を要する限界も有する.



% そのため,開発者は対象システムごとにSDCSとCDCSのどちらがより有効かを適切に判断する必要がある.
% しかし,現状では各手法で生成される複雑な中間生成LTSをシステムごとに手作業で把握し,どちらを適用するか判断するほかなく,実運用上の障壁となっている.
% 本論文では,このようなSDCSとCDCSの適用選択に関する課題を解消することを目的とする.

\section{CSDCS(CDCSとSDCSの一元化)}
\label{section:proposal}
CDCSとSDCSにおける適用選択の課題を解消するため,本論文ではSDCSとCDCSを一元化したDCS,Consolidated Stepwise Discrete Controller Synthesis(CSDCS)を提案する.
CSDCSの導入により,従来必要であったSDCSとCDCSの使い分けに関する判断工程が不要となり,両者の適用選択における課題を解消できる.
CSDCSではSDCSとCDCSの一元化を実現するため,\ref{subsection:limitation}節の制約1と制約2を克服するようにSDCSとCDCSを組み合わせる.

\subsection{技術詳細}
CSDCSでは,SDCS(Algorithm\ref{algorithm:SDCS})を基盤にしつつ,$g^{*}$の構築に関わる3つの過程(Parallel Composition,Modified Parallel Composition,Error State Abstraction)
をCDCSで導入されたConsolidated Game Composition(定義\ref{def:consolidated_game_composition})へと置換する.
CDCSは$g^{*}$の構築過程をConsolidated Game Compositionに統合することで$s_{err}$の抽象化を通じた計算空間削減を実現する.
このConsolidated Game Compositionは,SDCSにおいて繰り返し行われるゲーム空間$g^{*}$の構築処理にも適用可能であり,
SDCSの段階的な計算空間削減アプローチと整合的に機能する.
SDCSの構造にCDCSの手法を統合することにより,CDCS,SDCS,両手法の削減可能対象を包含し,CSDCSは削減可能対象の不一致に起因する制約1を克服する.
そのCSDCSアルゴリズムがAlgorithm\ref{algorithm:CSDCS}である.

\begin{algorithm}[h]
\caption{CSDCS}
\label{algorithm:CSDCS}
\begin{algorithmic}[1]
\renewcommand{\algorithmicrequire}{\textbf{Input:}}
\renewcommand{\algorithmicensure}{\textbf{Output:}}
\REQUIRE $E$ {\bf such that} $\forall e \in E, e = (id, S, A, \Delta, s^0)$
\REQUIRE $R$ {\bf such that} $\forall r \in R, r = (id, S, A, \Delta, s^0)$
\ENSURE $c$
\STATE $\Lambda =$ {\bf Algorithm \ref{algorithm:CSDCS_huristic}($E$, $R$)}
\FOR {$i^* = 1$ to $n(\Lambda)$}
    \STATE $\lambda = (i^*, eids_{\lambda}, rids_{\lambda}, cid_{\lambda}) \in \Lambda$
    \STATE $R^* = \{\forall r \in R \mid id_{r} \in rids_{\lambda}\}$
    \STATE $E^* = \{\forall e \in E \mid A_{e} \cap A_{R^*} \neq \emptyset\}$
    \STATE $g^* =$ {\bf Consolidated Game Composition($R^* \cup E^*$)}
    \STATE $c   =$ {\bf Safety Game Solving($g^*$)}
    \STATE $c^* = (cid_{\lambda}, S_{c}, A_{c}, \Delta_{c}, s^0_{c})$
    \STATE $R   = R \backslash R^*$, $E = E \backslash E^* \cup \{c^*\}$
\ENDFOR
\STATE {\bf return} $c$
\end{algorithmic}
\end{algorithm}

Algorithm\ref{algorithm:CSDCS}では,はじめに,任意のヒューリスティクスに基づいて合成シーケンス$\Lambda$を導出する(1行目).
その後,導出された$\Lambda$において$i_{\lambda}$が1の$\lambda$から順に繰り返しConsolidated Game Compositionを用いた部分合成(3-9行目)を行う.部分合成では,入力された$R$のうち$rids_{\lambda}$に$id_{r}$が含まれる$r$の集合を$R^*$とし(4行目),$R^*$と$E$から部分合成の適用可能条件式$A_{r} \subseteq A_{E^*} \backslash A_{E \backslash E^*}$を満たす$e \in E$の集合を$E^*$として導出する(5行目).その後,導出された$R^*$と$E^*$を入力としてConsolidated Game Compositionを実行することでゲーム空間$g^*$を合成する(6行目).その後,$g^*$を入力としてSafety Game Solvingを実行することで部分制御器$c$を合成し(7行目),合成された$c$の$id$に$cid_{\lambda}$を割り当てる(8行目).最後に,次の$\lambda$で使用する$R$と$E$から使用した$R^*$と$E^*$を取り除き,$cid_{\lambda}$が割り当てられた部分制御器$c^*$を$E$に加える(9行目).この部分合成を全ての$\Lambda$において$i_{\lambda}$の小さい方から順に繰り返し適用し,制御器$c$を合成する.

このとき,Algorithm\ref{algorithm:CSDCS}は,各監視モデルに対して個別に繰り返し部分合成を実施するため,SDCSと同様にbasic DCSと比較して計算時間が増加する可能性(制約2)を内在する.
この計算時間の増加に対処するため,CSDCSでは複数の監視モデルを対象とした部分合成手法を導入し,部分合成の繰り返し回数を削減する.

SDCSは,部分合成の回数を減少させるほど,各回で構築される状態空間が大規模化するため,計算空間削減性能が低下する.
対して,Consolidated Game Compositionは,同時に入力される監視モデルの数が増加するにつれて,抽象化される違反状態に該当する状態が増え,計算空間削減性能が向上する傾向を持つ.
CSDCSでは,Consolidated Game Compositionのこの傾向を活用する.
複数の監視モデルを対象とした部分合成にConsolidated Game Compositionを適用することで,SDCS特有の計算空間削減効果を損なうことなく計算時間の増加を抑制し,制約2を克服する.
複数の監視モデルを同時に部分合成するため,CSDCSではAlgorithm\ref{algorithm:CSDCS}における$\Lambda$導出過程(1行目)を改良する.
$\lambda \in \Lambda$の$rids_{\lambda}$に複数の監視モデルの$id$を格納することで,同一ステップ内で複数の監視モデルを同時に対象とする部分合成が実現される.
そのCSDCSのための$\Lambda$導出アルゴリズムがAlgorithm\ref{algorithm:CSDCS_huristic}である.

\begin{algorithm}[h]
\caption{CSDCSに特化した$\Lambda$の統合}
\label{algorithm:CSDCS_huristic}
\begin{algorithmic}[1]
\renewcommand{\algorithmicrequire}{\textbf{Input:}}
\renewcommand{\algorithmicensure}{\textbf{Output:}}
\REQUIRE $E$ {\bf such that} $\forall e \in E, e = (id_{e}, S_{e}, A_{e}, \Delta_{e}, s^0_{e})$
\REQUIRE $R$ {\bf such that} $\forall r \in R, r = (id_{r}, S_{r}, A_{r}, \Delta_{r}, s^0_{r})$
\ENSURE  $\Lambda$
\STATE $\Lambda =$ {\bf Algorithm \ref{algorithm:huristic}($E$, $R$, $1$)}, \;\;$i^{**} = 1$
\STATE $cids = \{\forall cid_{\lambda} \mid (i_{\lambda}, eids_{\lambda}, rids_{\lambda}, cid_{\lambda}) \in \Lambda\}$
\FOR {$i^* = 1$ to $n(\Lambda)$}
    \STATE $\lambda = (i^*, eids_{\lambda}, rids_{\lambda}, cid_{\lambda}) \in \Lambda$
    \IF{$n(eids_{\lambda} \cap cids) = 1$}
        \STATE $cid^* = \exists id \in (eids_{\lambda} \cap cids)$
        \STATE $\lambda^* = (i_{\lambda^*}, eids_{\lambda^*}, rids_{\lambda^*}, cid^*) \in \Lambda$
        \STATE $eids^* = eids_{\lambda} \cup eids_{\lambda^*} \backslash \{cid^*\}$
        \STATE $rids^* = rids_{\lambda} \cup rids_{\lambda^*}$
        \STATE $\Lambda = \Lambda \backslash \{\lambda ,\lambda^*\} \cup \{i_{\lambda^*}, eids^*, rids^*, cid_{\lambda}\}$
    \ELSE
        \STATE $\Lambda = \Lambda \backslash \{\lambda\} \cup \{i^{**}, eids_{\lambda}, rids_{\lambda}, cid_{\lambda}\}$
        \STATE $i^{**} = i^{**} + 1$
    \ENDIF
\ENDFOR
\STATE {\bf return} $\Lambda$
\end{algorithmic}
\end{algorithm}

Algorithm\ref{algorithm:CSDCS_huristic}では,Algorithm\ref{algorithm:huristic}を用いて監視モデルごとに部分合成する$\Lambda$を導出する(1行目).その後,$\Lambda$で合成される全ての部分制御器の$cid$を$cids$として用意し(2行目),$i_{\lambda}$が1の$\lambda \in \Lambda$から順に$eids_{\lambda}$に$cid$がいくつ含まれるか確認する(3-15行目).
$eids_{\lambda}$に$cid$が1つ含まれる時(5行目),その$cid$を合成する$\lambda^*$と$\lambda$を統合し,$\{i_{\lambda^*}, eids^*, rids^*, cid_{\lambda}\}$を構築する(6-10行目).
$eids^*$は$eids_{\lambda}$に含まれる$cid_{\lambda^*}$と$eid_{\lambda^*}$を置き換えた$id$の集合(8行目),
$rids^*$は$rids_{\lambda}$と$rid_{\lambda^*}$の和集合(9行目)となる.
その後,統合された$\{i^{**}, eids_{\lambda}, rids_{\lambda}, cid_{\lambda}\}$と,統合元となった$\lambda$と$\lambda^*$を,$\Lambda$において置き換える(10行目).
$eids_{\lambda}$に$cid$が2つ以上含まれる,もしくは1つも含まれない場合は,統合で減った$\lambda$の数に合わせて$i_{\lambda}$を振り直す(11-14行目).
以上の工程を全ての$\lambda \in \Lambda$において繰り返し実行し,$n(eids_{\lambda} \cap cids)$が1である$\lambda$の統合を行う.

このとき,$n(eids_{\lambda} \cap cids)$は,$\lambda$の部分合成で環境モデル$E^*$として入力される部分制御器(以前の部分合成で合成された$c^*$)の数を表す.
部分制御器には,その部分制御器の合成過程で安全性を違反すると判明した部分違反状態は含まれない.
部分制御器を$E^*$に含む部分合成では,$g^*$構築時であっても,部分制御器の部分違反状態は重ね合わされないため,構築回避され,状態削減される.
すると,複数の部分制御器を$E^*$に含む部分合成ほど,重ね合わせにより,削減される状態数は指数関数的に増加するため,Algorithm\ref{algorithm:CSDCS_huristic}において,$\lambda$の部分合成で削減される状態数は$n(eids_{\lambda} \cap cids)$に対して指数関数的に増加する傾向がある.

そこで,CSDCSでは,この$n(eids_{\lambda} \cap cids)$の数に着目し,部分合成による計算空間削減効果が特に大きくなる$n(eids_{\lambda} \cap cids)$が2以上の時のみ部分合成を適用する.そして,比較的効果の小さい$n(eids_{\lambda} \cap cids)$が1の時はConsolidated Game Compositionが効果的に働くように複数の監視モデルを同時に部分合成することで,高い計算空間削減効果を維持したまま,Algorithm\ref{algorithm:CSDCS_huristic}における部分合成回数を抑制した.


\subsection{機能の妥当性証明}
機能の妥当性を示すため,CSDCSが常にbasic DCSと同じ制御器を合成できることを,命題\ref{prf:proof1}を通して証明する.

\begin{pro}
    \label{prf:proof1}
    1以上の自然数$i$と$j$において,環境モデル$E = \{e_1, e_2, \ldots, e_i\}$,監視モデル$R = \{r_1, r_2, \ldots, r_j\}$が与えられるとき,{\bf Algorithm \ref{algorithm:basic_controller_synthesis}($E$, $R$)}の出力$c$と,{\bf Algorithm \ref{algorithm:CSDCS}($E$, $R$)}の出力$c$は同じである.
\end{pro}
\begin{proof}
    % 本証明では,同じ$E$と$R$が与えられるとき,Algorithm\ref{algorithm:CSDCS}(CSDCS)とAlgorithm\ref{algorithm:SDCS}(SDCS)が同じ$c$を合成することを示す.そして,Algorithm\ref{algorithm:SDCS}(SDCS)がAlgorithm\ref{algorithm:basic_controller_synthesis}(basic DCS)と同じ$c$を合成することを示すことで,命題\ref{prf:proof1}が成り立つことを証明する.
    同じ$E$と$R$が与えられるとき,Algorithm\ref{algorithm:SDCS}(SDCS)とAlgorithm\ref{algorithm:basic_controller_synthesis}(basic DCS)は同じ$c$を合成する\cite{yamauchi:IEICEJ2023}.よって,式\ref{equation:SDCS_basicDCS}が成り立つ.
    \begin{equation}
    \label{equation:SDCS_basicDCS}
        {\bf Algorithm \ref{algorithm:SDCS}(E, R)} = {\bf Algorithm \ref{algorithm:basic_controller_synthesis}(E, R)}
    \end{equation}

    次に,Algorithm\ref{algorithm:SDCS}(SDCS)とAlgorithm\ref{algorithm:CSDCS}(CSDCS)は同じ$c$を合成することを証明する.
    Algorithm\ref{algorithm:CSDCS}では,Consolidated Game Composition,Algorithm\ref{algorithm:SDCS}ではParallel Composition,Modified Parallel Composition,Error State Abstractionの3つを用いて$g^*$を合成する.
    同じ$E^*$と$R^*$が与えられるとき,Consolidated Game Compositionによって合成される$g^*$と,Parallel Composition,Modified Parallel Composition,Error State Abstractionの3つを通して合成される$g^*$は同じである\cite{yamauchi:IPSJ2024}.
    よって,Algorithm\ref{algorithm:CSDCS}とAlgorithm\ref{algorithm:SDCS}で合成される$g^*$は同じである.
    すると,$g^*$から$c$を導出する過程は同じであるため,同じ$E$と$R$が与えられるときAlgorithm\ref{algorithm:CSDCS}とAlgorithm\ref{algorithm:SDCS}は同じ$c$を合成する.
    よって,式\ref{equation:CSDCS_SDCS}が成り立つ.
    \begin{equation}
    \label{equation:CSDCS_SDCS}
        {\bf Algorithm \ref{algorithm:CSDCS}(E, R)} = {\bf Algorithm \ref{algorithm:SDCS}(E, R)}
    \end{equation}

    すると,式\ref{equation:CSDCS_SDCS}と式\ref{equation:SDCS_basicDCS}から式\ref{equation:CDDCS_SDCS_basicDCS}が成り立つ.
    \begin{align}
    \label{equation:CDDCS_SDCS_basicDCS}
        \begin{split}
            {\bf Algorithm \ref{algorithm:CSDCS}(E, R)} &= {\bf Algorithm \ref{algorithm:SDCS}(E, R)}\\
            &= {\bf Algorithm \ref{algorithm:basic_controller_synthesis}(E, R)}
        \end{split}
    \end{align}

    式\ref{equation:CDDCS_SDCS_basicDCS}より,同じ$E$と$R$が与えられるとき,Algorithm\ref{algorithm:CSDCS}とAlgorithm\ref{algorithm:basic_controller_synthesis}で合成される$c$は同じであるため,本命題は成り立つ.
\end{proof}


% [証明2]huristicが効果的であることの証明
% どのような計算空間を削減するか
% 無理そうな気もするので,それは実験で評価した方が良いかも









% 以下,監視対象モデル数最小のSDCSを導出する話
% ---------------------------------------------------------------




% \subsection{CSDCSのための合成シーケンス導出アルゴリズム}
% \label{subsection:CSDCS_huristic}
% CSDCSでは,新たに導入されたConsolidated Game Compositionによる計算空間削減効果を最大限発揮させるため,複数の監視モデルを同時に分析(部分合成)することが求められる.
% しかし,ベースとなっているSDCSは可能な限り監視モデルごとに細かく分析(部分合成)するほど計算空間削減効果が大きくなる手法であるため,
% 両手法を取り入れたCSDCSの計算空間削減効果を最大化する上で,どの監視モデルを一括で分析(部分合成)し,どの監視モデルを分割して分析(部分合成)するか$\Lambda$の設計(Algorithm\ref{algorithm:CSDCS} 1行目)において重要となる.
% そこで,本論文ではCSDCSの提案にあたって$E$と$R$の監視関係に着目した汎用的な$\Lambda$導出アルゴリズムを提供する.
% その$\Lambda$導出アルゴリズムがAlgorithm\ref{algorithm:CSDCS_huristic}である.

% % 合成プロセスの統合
% \begin{algorithm}[t]
% \caption{Consolidated Game Compositionを考慮した$\Lambda$導出}
% \label{algorithm:CSDCS_huristic}
% \begin{algorithmic}[1]
% \renewcommand{\algorithmicrequire}{\textbf{Input:}}
% \renewcommand{\algorithmicensure}{\textbf{Output:}}
% \REQUIRE $E$ {\bf such that} $\forall e \in E, e = (id_{e}, S_{e}, A_{e}, \Delta_{e}, s^0_{e})$
% \REQUIRE $R$ {\bf such that} $\forall r \in R, r = (id_{r}, S_{r}, A_{r}, \Delta_{r}, s^0_{r})$
% \ENSURE  $\Lambda$
% \STATE $\Lambda =$ {\bf Algorithm \ref{algorithm:influence} ($E$, $R$, $1$)}
% \STATE $cids = \{\forall cid_{\lambda} \mid (step_{\lambda}, eids_{\lambda}, rids_{\lambda}, cid_{\lambda}) \in \Lambda\}$
% \FOR {$i = 1$ to $n(\Lambda)$}
%     \STATE $\lambda = (i, eids_{\lambda}, rids_{\lambda}, cid_{\lambda}) \in \Lambda$
%     \IF{$n(eids_{\lambda} \cap cids) = 1$}
%         \STATE $cid' = \exists id \in (eids_{\lambda} \cap cids)$
%         \STATE $\lambda' = (step_{\lambda'}, eids_{\lambda'}, rids_{\lambda'}, cid') \in \Lambda$
%         \STATE $eids^* = eids_{\lambda} \cup eids_{\lambda'} \backslash \{cid'\}$
%         \STATE $rids^* = rids_{\lambda} \cup rids_{\lambda'}$
%         \STATE $\Lambda = \Lambda \backslash \{\lambda ,\lambda'\} \cup \{step_{\lambda'}, eids^*, rids^*, cid_{\lambda}\}$
%     \ENDIF
% \ENDFOR
% \STATE {\bf return} $\Lambda$
% \end{algorithmic}
% \end{algorithm}

% Algorithm\ref{algorithm:CSDCS_huristic}では,はじめにCSDCSで分析される$R$全体の監視対象モデル数が最小となるような合成シーケンス$\Lambda$を導出する(1行目).その後,導出された合成プロセス$\lambda$の$\Lambda$における依存関係に応じて,複数の$\lambda$を統合する(2-12行目).
% $\Lambda$に含まれる全ての部分制御器のid$cid_{\lambda}$の集合を$cids$として用意し,$step_{\lambda}$が1の$\lambda \in \Lambda$から順に$eids_{\lambda}$に部分制御器のidが何個含まれているか確認する(3-5行目).$eids_{\lambda}$に含まれる部分制御器のidが1つの時(5行目),その部分制御器のidを$cid'$として取り出し(6行目),$cid'$を合成する$\lambda$を$\lambda'$として用意する(7行目).そして,$eids_{\lambda}$に含まれるidから$cid'$を除き,その$cid'$を合成する際に使用する環境モデルのid$eid_{\lambda'}$を加えたものを$eids^*$(8行目).$rids_{\lambda}$に含まれるidに$cid'$を合成する際に使用する監視モデルのid$rid_{\lambda'}$を加えたものを$rids^*$とする(9行目).最後に,用意した$eids^*$と$rids^*$から$\lambda$と$\lambda$を統合した合成プロセスを構築し,$\Lambda$において$\lambda$,$\lambda'$と置き換える(10行目).以上の工程を全ての$\lambda \in \Lambda$において繰り返し実行し,合成プロセスの統合を行う.このとき,Algorithm\ref{algorithm:CSDCS_huristic}の1行目にて導出されるCSDCSで分析される$R$全体の監視対象モデル数が最小となるような合成シーケンス$\Lambda$は,Algorithm\ref{algorithm:influence}にて導出される.
% % [ToDo] なぜこの統合によってConsolidated Game Compositionの計算削減効果が発揮されるのか?(例を使った方が良い?)


% % 影響量を考慮した合成プロセスの構築
% \begin{algorithm}[t]
% \caption{部分合成影響量を考慮した$\Lambda$の導出}
% \label{algorithm:influence}
% \begin{algorithmic}[1]
% \renewcommand{\algorithmicrequire}{\textbf{Input:}}
% \renewcommand{\algorithmicensure}{\textbf{Output:}}
% \REQUIRE $E$ {\bf such that} $\forall e \in E, e = (id_{e}, S_{e}, A_{e}, \Delta_{e}, s^0_{e})$
% \REQUIRE $R$ {\bf such that} $\forall r \in R, r = (id_{r}, S_{r}, A_{r}, \Delta_{r}, s^0_{r})$, $step^*$
% \ENSURE  $\Lambda$
% \STATE $\delta^{\min} = \infty$,\;\; $cid^* = ${\bf uniqueID}
% \FORALL{$r \in R$}
%     \STATE $E^* = \{\forall e \in E \mid A_{e} \cap A_{r} \neq \emptyset\}$
%     \STATE $E^\delta = E \backslash E^* \cup \bigcup_{e \in E} \{(id_{e}, \emptyset, A_{E^*}, \emptyset, \emptyset)\}$
%     \STATE $\delta = \sum_{r \in R} \big(\mu(r, E^{\delta}) - \mu(r, E)\big)$
%     \IF{$\delta < \delta^{\min}$}
%         \STATE $\delta^{\min} = \delta$, \;\;$R^* = \emptyset$
%         \FORALL {$r' \in R \backslash \{r\}$}
%             \STATE $E_{r'} = \{\forall e \in E \mid A_{e} \cap A_{r'} \neq \emptyset\}$
%             \IF {$E_{r'} \subseteq E_{r}$}
%                 \STATE $R^* = R^* \cup \{r'\}$
%             \ENDIF
%         \ENDFOR
%         \STATE $eids^* = \{\forall id_{e} \mid (id_{e}, S_{e}, A_{e}, \Delta_{e}, s^0_{e}) \in E^*\}\}$
%         \STATE $rids^* = \{\forall id_{r} \mid (id_{r}, S_{r}, A_{r}, \Delta_{r}, s^0_{r}) \in R^*\}\}$
%         \STATE $c^* = (cid^*, \emptyset, A_{E^*}, \emptyset, \emptyset)$
%         \STATE $E' = E \backslash E^* \cup \{c^*\}$, $R' = R \backslash R^*$
%     \ENDIF
% \ENDFOR
% \STATE {\bf return} $\Lambda = \{(step^*, eids^*, rids^*, cid^*)\}$ $\cup$ {\bf Algorithm \ref{algorithm:influence}($E'$, $R'$, $step^*+1$)}
% \end{algorithmic}
% \end{algorithm}

% Algorithm\ref{algorithm:CSDCS_huristic}では,監視モデルの監視対象モデル数に着目し,$E$と$R$から全監視モデルの監視対象モデル数が最小だと思われる合成シーケンス$\Lambda$を導出する.
% このとき,監視対象モデル数は定義\ref{def:monitored_model}に基づいて導出できる.

% \begin{dfn}{\textbf{監視対象モデル数 }}
% \label{def:monitored_model}
%     監視対象モデル数$\mu(r, E)$は,監視モデル$r$と環境モデルの集合$E$において部分合成の適用可能条件$A_{r} \subseteq A_{E^*} \backslash A_{E \backslash E^*}$を満たす$E^*$の最小の要素数を表す.
%     この時,要素数最小の$E^*$は$\{\forall e \in E \mid A_e \cap A_r \neq \emptyset\}$で導出できるため,その要素数である$\mu(r, E)$は以下の式によって導出できる.
%     \begin{equation}
%     \label{function:cost}
%         \mu \langle r,E \rangle = n(\{\forall e \in E \mid A_e \cap A_r \neq \emptyset\})
%     \end{equation}
% \end{dfn}

% 監視対象モデル数$\mu \langle r,E \rangle$は,$r$の部分合成で構築されるゲーム空間$g^*$の構築要素となる環境モデルの数を表し,監視対象モデル数に対して$g^*$の状態数は指数関数的に増加する性質を持つ.部分合成を用いたDCSでは後に分析される$r$の$g^*$ほど空間削減効果が大きくなるため,監視対象モデル数が大きなモデルほど後に分析することで,部分合成の計算空間削減効果が大きくなると考えられる.
% よって,監視対象モデル数が最小のものから順に部分合成する$\Lambda$を用いることで計算空間削減効果を大きくできるかと考えられるが,その$r$の監視対象モデル数がいくら小さくとも,$r$以外の監視対象モデル数を増加させる$r$も存在する.
% そこで,その状況に対処するために$r$を部分合成することによる$r$以外の監視対象モデル数の増加量を$r$と部分合成影響量$\delta$とし,この$\delta$を考慮することで,$E$と$R$から全監視モデルの監視対象モデル数が最小だと思われる合成シーケンス$\Lambda$が導出できると考えた.
% このとき,部分合成影響量$\delta$は以下にように定義する.

% \begin{dfn}{\textbf{部分合成影響量 }}
% \label{def:component_model}
%     部分合成影響量$\delta$は,監視モデル$r$を部分合成する前の環境モデル群を$E$,$r$を部分合成する前の環境モデル群を$E^\delta$とした時,全ての$R$において
%     % 影響量$\delta$は,SDCSにおいて$r$を部分合成した時監視モデル全体の監視対象モデル数が
%     \begin{equation}
%     \label{function:influence}
%         \delta = \sum_{r \in R} \big(\mu \langle r, E^{\delta} \rangle - \mu \langle r, E \rangle\big)
%     \end{equation}
% \end{dfn}


% 計算空間効率の良い合成プロセスを導出するにあたって,部分制御器が未分析な監視モデルの監視対象モデルとなるSDCSの仕組みに着目した.
% SDCSでは,入力の監視モデルを順序づけし,その順序に従ってゲーム空間を構築・分析していく.
% そしてゲーム空間の分析する度に部分制御器が出力され,以降に分析される監視モデルの監視対象モデルとなる.

% この時,〜の例を考えてみる.
% SDCSにおいてはじめに分析する監視モデルの監視対象モデルが全コンポーネントモデルとなる場合,以降全ての監視モデルにおいて全コンポーネントモデルで分析を行う必要があるため,非常に効率が悪い.

% よって,未分析の監視モデルの監視対象モデル数を考慮することで,制御器を合成するにあたって構築される最大のゲーム空間を小さくすることが可能となる.

% そこで本論文では,「監視モデルの分析によって,未分析な監視モデルの監視対象モデル数がどれだけ増加するか」を表した指標である,影響量を考慮して計算空間効率の良い合成プロセスを構築する.
% この時,監視対象モデル数は定義\ref{def:monitored_model}によって定義される.

% \begin{dfn}{\textbf{監視対象モデル数 }}
% \label{def:num_of_monitored_model}
%     監視対象モデル数$\mu(r, E)$は,監視モデル$r$と環境モデルの集合$E$において部分合成の適用可能条件$A_{r} \subseteq A_{E^*} \backslash A_{E \backslash E^*}$を満たす$E^*$の最小の要素数を表す.
%     この時,要素数最小の$E^*$は$\{\forall e \in E \mid A_e \cap A_r \neq \emptyset\}$で導出できるため,その要素数である$\mu(r, E)$は以下の式によって導出できる.
%     \begin{equation}
%     \label{function:cost}
%         \mu \langle r,E \rangle = n(\{\forall e \in E \mid A_e \cap A_r \neq \emptyset\})
%     \end{equation}
% \end{dfn}



% この監視対象モデル数を用いて,影響量は定義\ref{def:component_model}で定義される.

% \begin{dfn}{\textbf{影響量 }}
% \label{def:component_model}
%     影響量$\delta$は,監視モデル$r$を部分合成する前の環境モデル群を$E$,$r$を部分合成する前の環境モデル群を$E^\delta$とした時,全ての$R$において
%     % 影響量$\delta$は,SDCSにおいて$r$を部分合成した時監視モデル全体の監視対象モデル数が
%     \begin{equation}
%     \label{function:influence}
%         \delta = \sum_{r \in R} \big(\mu \langle r, E^{\delta} \rangle - \mu \langle r, E \rangle\big)
%     \end{equation}
% \end{dfn}

% この影響量が最小となる監視モデルから優先的に分析する合成プロセスを適用することで,効率的な計算空間で制御器の合成が可能となる.

% % 一括分析を考慮した合成プロセスの統合
% 分析対象モデルに部分制御器を一つしか含まない場合,合成を集約する.



% その複数の
% Algorithm\ref{algorithm:SDCS}における計算時間増加の課題に対処するために,CSDCSでは最小限の部分合成回数で高い計算空間削減効果が期待される合成シーケンスであることが望ましい.

% そこでCSDCSでは,可能な限り複数の監視モデルを同時に部分合成し,計算空間がボトルネックとなる可能性が高いタイミングでのみ部分合成を適用する,CSDCSに特化した合成シーケンスを提案する.

% 計算空間がボトルネックとなりやすいタイミングとしては,まずはじめに複数の環境モデルを入力とした部分合成が実施されるタイミングが考えられる.
% 部分合成過程で行われるConsolidated Game Compositionの性質から,入力の環境モデル数に対して出力のゲーム空間の状態数は指数増加する.
% よって環境モデル数が増加する監視対象モデル数が二以上の部分合成はボトルネックとなりやすい.

% しかし,この条件だけでは最悪で環境モデルの数だけ部分合成する必要があるため,まだ部分合成の数が多い.
% そこで,

% よって,本論文では部分制御器を2つ以上含む部分合成で入力となる部分制御器の部分合成だけ実施することで,少ない部分合成回数で高い計算空間削減効果を目指す.
% そのアルゴリズムがAlgorithm\ref{algorithm:CSDCS_huristic}である.



% 部分合成する回数を減らすにあたって,部分的に分析した方が良い監視モデルと,他と同時に分析しても良い監視モデルを考える必要があるが,本論文では2つの監視モデルの監視対象モデルの依存関係に着目した.




% 監視対象モデルが完全に被らない場合→分解して解くべき(SDCSが有効に働く)\\
% 監視対象モデルに重複が見られる場合→同じサブ問題として解くべき(CDCSが有効に働く)\\

% 一部の監視モデルを同時に分析するにあたって,同時に分析する監視モデルの数が多くなるほどConsolidated Game Compositionの計算空間削減効果が大きくなる性質を活用する.


% -----
% また,Consolidated Game Compositionは複数の監視モデルを同時に分析するほど計算空間を削減できる手法であるため,監視モデルごとに分析していてはCDCSより削減効果が小さくなる.

% そこでCSDCSでは,一部の監視モデルを同時に部分合成する.
% 一部の監視モデルを同時に分析することで,SDCSの計算空間削減効果は小さくなるが,Consolidated Game Compositionの削減効果は大きくなる.また,監視モデルを同時に分析するほど,CSDCSにおける部分合成の回数が少なくなるため計算時間の増加抑制することが可能となる.
% そこで,続く\ref{subsection:CSDCS_2}節では,一部の監視モデルを同時に部分合成するCSDCSに特化した合成シーケンス導出アルゴリズムを提案する.
% -----

% Algorithm\ref{algorithm:CSDCS}の1行目において,SDCSで用いられてきた監視モデルごとに部分合成する合成シーケンスを用いると,CDCSの計算空間削減効果が十分に発揮されない場合が存在する.

% 式\ref{function:SDCS_reduced_state}より,SDCSは分析済みの監視モデル数が多くなるほど,違反状態に該当する状態数が増加するため計算空間削減効果が大きくなる.
% そこで,SDCSではどのゲーム空間がSDCSのボトルネックになったとしても分析済みの監視モデル数が最大化されるよう,一度の分析(部分合成)で分析する監視モデル数が最小となるような監視モデルごとにゲーム空間を構築する合成シーケンスが提案されてきた.

% しかし,式\ref{function:CDCS_reduced_state}より,CSDCSで導入するConsolidated Game Compositionは,一度の分析(部分合成)で分析する監視モデルの数が多くなるほど,違反状態に該当する状態空間が多くなるため計算空間削減効果が大きくなる.
% このことから,SDCSとConsolidated Game Compositionの両方の計算空間削減効果を最大化するための,CSDCSに特化した合成シーケンス導出ポリシーが必要となる.

% そこで本論文では,CSDCSで構築されるゲーム空間の包含関係に着目した,CSDCSに特化した合成シーケンスを提案する.
% 監視モデルごとに部分合成する従来の合成シーケンスに対して,SDCSによる計算空間削減効果が小さいと思われる合成プロセスを統合することで,CSDCSにおいてConsolidated Game Compositionの計算空間削減効果を最大化する.
% このとき,統合する合成プロセスはゲーム空間を構成する環境モデルの要素に着目して決定する.

%[ToDo] なんで部分制御器がひとつの場合合成プロセスを統合するのかの理由がほしい


% すると,統合される合成プロセスは常に以下の条件式\ref{function:merge_process}を満たす.
% \begin{equation}
% \label{function:merge_process}
%     n(eids_{\lambda} \cap cids) = 1
% \end{equation}


% % ゲーム空間の包含関係に着目した削減効果に関する説明

% 以上の合成プロセスの統合技術を導入した,CSDCSに特化した合成シーケンス$\Lambda$導出アルゴリズムがAlgorithm\ref{algorithm:CSDCS_huristic}である.

% % 合成プロセスの統合
% \begin{algorithm}[t]
% \caption{Consolidated Game Compositionを考慮した$\Lambda$導出}
% \label{algorithm:CSDCS_huristic}
% \begin{algorithmic}[1]
% \renewcommand{\algorithmicrequire}{\textbf{Input:}}
% \renewcommand{\algorithmicensure}{\textbf{Output:}}
% \REQUIRE $E$ {\bf such that} $\forall e \in E, e = (id_{e}, S_{e}, A_{e}, \Delta_{e}, s^0_{e})$
% \REQUIRE $R$ {\bf such that} $\forall r \in R, r = (id_{r}, S_{r}, A_{r}, \Delta_{r}, s^0_{r})$
% \ENSURE  $\Lambda$
% \STATE $\Lambda =$ {\bf Huristic($E$, $R$)}
% \STATE $cids = \{\forall cid_{\lambda} \mid (step_{\lambda}, eids_{\lambda}, rids_{\lambda}, cid_{\lambda}) \in \Lambda\}$
% \FOR {$i = 1$ to $n(\Lambda)$}
%     \STATE $\lambda = (i, eids_{\lambda}, rids_{\lambda}, cid_{\lambda}) \in \Lambda$
%     \IF{$n(eids_{\lambda} \cap cids) = 1$}
%         \STATE $cid' = \exists id \in (eids_{\lambda} \cap cids)$
%         \STATE $\lambda' = (step_{\lambda'}, eids_{\lambda'}, rids_{\lambda'}, cid') \in \Lambda$
%         \STATE $eids^* = eids_{\lambda} \cup eids_{\lambda'} \backslash \{cid'\}$
%         \STATE $rids^* = rids_{\lambda} \cup rids_{\lambda'}$
%         \STATE $\Lambda = \Lambda \backslash \{\lambda ,\lambda'\} \cup \{step_{\lambda'}, eids^*, rids^*, cid_{\lambda}\}$
%     \ENDIF
% \ENDFOR
% \STATE {\bf return} $\Lambda$
% \end{algorithmic}
% \end{algorithm}

% % アルゴリズムの詳細
% Algorithm\ref{algorithm:CSDCS_huristic}では,はじめにCSDCSで分析される$R$全体の監視対象モデル数が最小となるような合成シーケンス$\Lambda$を導出する(1行目).その後,導出された合成プロセス$\lambda$の$\Lambda$における依存関係に応じて,複数の$\lambda$を統合する(2-12行目).
% $\Lambda$に含まれる全ての部分制御器のid$cid_{\lambda}$の集合を$cids$として用意し,$step_{\lambda}$が1の$\lambda \in \Lambda$から順に$eids_{\lambda}$に部分制御器のidが何個含まれているか確認する(3-5行目).$eids_{\lambda}$に含まれる部分制御器のidが1つの時(5行目),その部分制御器のidを$cid'$として取り出し(6行目),$cid'$を合成する$\lambda$を$\lambda'$として用意する(7行目).そして,$eids_{\lambda}$に含まれるidから$cid'$を除き,その$cid'$を合成する際に使用する環境モデルのid$eid_{\lambda'}$を加えたものを$eids^*$(8行目).$rids_{\lambda}$に含まれるidに$cid'$を合成する際に使用する監視モデルのid$rid_{\lambda'}$を加えたものを$rids^*$とする(9行目).最後に,用意した$eids^*$と$rids^*$から$\lambda$と$\lambda$を統合した合成プロセスを構築し,$\Lambda$において$\lambda$,$\lambda'$と置き換える(10行目).以上の工程を全ての$\lambda \in \Lambda$において繰り返し実行し,合成プロセスの統合を行う.
% % このとき,Algorithm\ref{algorithm:CSDCS_huristic}の1行目にて導出されるCSDCSで分析される$R$全体の監視対象モデル数が最小となるような合成シーケンス$\Lambda$は,Algorithm\ref{algorithm:influence}にて導出される.


\section{評価}
\label{section:evaluation}
CSDCSの有効性を評価するために,本論文では以下のResearch Question(RQ)を設定し評価実験を行う.

{
\begin{description}
  \item[RQ1] CSDCSはどれほどの削減効果を持つか
  \item[RQ2] CDCSと同等以上の削減効果を持つか
  \item[RQ3] SDCSと同等以上の削減効果を持つか
  \item[RQ4] SDCSとCDCSの適用選択の課題を解消できるか
\end{description}
}

RQ1では,basic DCSを通してCSDCSの各削減効果とその傾向を評価する.
RQ2とRQ3では,CSDCSの各削減効果がCDCSとSDCSと比較して優位であるか相対的に評価する.
RQ4では,CSDCSがSDCSとCDCSの適用選択の課題を解消しうるか評価する.

\subsection{実験設定}
CSDCSを評価するにあたって,本論文ではbasic DCS,SDCS,CDCS,CSDCSの4つのDCS手法で制御器の合成実験を実施し,その際に構築された最大計算空間の状態数$|S|$,遷移数$|\Delta|$,合成に必要となった必要主記憶量$M$,必要計算時間$T$を計測した.

また,環境モデルの数が変化するパラメータ$N$と環境モデルの状態数が変化するパラメータ$K$が操作できるスケーラブルな7つの離散事象システムを対象に制御器の合成実験を行った.
Headcount Control(HC)\cite{paper:ArtGallery}は,複数の部屋の人数を管理することを目的としたシステムであり,入室の許可/拒否を制御する.$N$には制御する部屋の数,$K$には管理可能なひと部屋あたりの最大収容人数が該当する.
Auto Warehouse(AW)\cite{paper:KIVA_System}は倉庫の荷物運搬業務の自動化を目的としたシステムであり,同じ機能を持つ荷物運搬ロボットを在庫管理や出荷作業など複数の役割に分けつつ協調制御する.
$N$には制御するロボットの数,$K$には在庫管理する棚の数が該当する.
Air Traffic(AT)\cite{paper:DES}は,空路と滑走路の混雑管理を目的としたシステムであり,複数の飛行機の離着陸,空路での待機,空路の移動を制御する.$N$には制御する飛行機の数,$K$には管理可能な空路の数が該当する.
Bidding Workflow(BW)\cite{paper:DES}は,申請書ごとの複雑な承認手続きの自動管理を目的としたシステムであり,受け取った申請書の提出先・提出順を制御する.
$N$は承認をおこなう提出先となる管理部署数,$K$は申請が否認された場合に再申請できる回数$K$が該当する.
Cat \& Mouse(CM)\cite{paper:DES}は,制限されたフィールド上で自由に動く捕獲対象の捕獲を目的とするシステムであり,捕獲ロボットの移動を制御する.$N$は捕獲ロボットの数,$K$は捕獲対象が逃げ回るフィールドの区画数が該当する.
Drone Control(DC)\cite{yamauchi:ICCSCE2023}\cite{yamauchi:SESOS2023}は,制限されたフィールドを巡回監視する警備ドローンの飛行管理を目的としたシステムであり,複数のドローンの移動,充電を制御する.$N$は管理するドローンの数,$K$はフィールドの区画数が該当する.
Access Control(AC)\cite{yamauchi:AIKE2020}は,複数サーバ内の情報閲覧をアクセス元に応じて管理することを目的としたシステムであり,管理サーバの情報開示処理,情報加工処理,応答処理を管理する.$N$も$K$もシステムが管理するサーバ数が該当し,本シナリオでは常に同じ値を取る.

実験には,OSがWindows 10 Pro (64bit),CPUがIntel Xeon W-2265,RAMが256GB(64GB×4)の計算機で,制御器合成ツール Modal Transition System Analyzer\cite{paper:MTSA}を使用した.
以上の実験設定で評価実験をおこなった結果が,表\ref{table:space}と表\ref{table:cost}である.

\begin{table*}[ht]
\scriptsize
\caption{DCSにおける最大の状態数と遷移数(削減率:\%)}
\ecaption{Maximum number of states and transitions in DCS (Reduction rate: \%)}
\label{table:space}{}
\begin{tabular}{c|cc|rr|rr|rr|rr}
\toprule
\multicolumn{3}{c|}{} &\multicolumn{2}{c|}{basic DCS} &\multicolumn{2}{c|}{CDCS} &\multicolumn{2}{c|}{SDCS} &\multicolumn{2}{c}{CSDCS}\\
\multicolumn{1}{c}{} &\multicolumn{1}{c}{N} &\multicolumn{1}{c|}{K}
&\multicolumn{1}{c}{$|S|$} &\multicolumn{1}{c|}{$|\Delta|$}
&\multicolumn{1}{c}{$|S|$} &\multicolumn{1}{c|}{$|\Delta|$}
&\multicolumn{1}{c}{$|S|$} &\multicolumn{1}{c|}{$|\Delta|$}
&\multicolumn{1}{c}{$|S|$} &\multicolumn{1}{c}{$|\Delta|$}\\
\hline 
\multirow{7}{*}{{\rotatebox[origin=c]{90}{HC}}}
&3 &4 &14.1 &61.6 &{\bf *0.66(-95.3)} &{\bf *2.96(-95.2)} &3.68(-73.9) &6.90(-88.8) &{\bf *0.66(-95.3)} &{\bf *2.96(-95.2)} \\
&4 &4 &277 &1514 &{\bf *4.07(-98.5)} &{\bf *22.5(-98.5)} &23.1(-91.7) &45.0(-97.0) &{\bf *4.07(-98.5)} &{\bf *22.5(-98.5)} \\
&5 &4 &2016 &12912 &{\bf *10.5(-99.5)} &{\bf *69.0(-99.5)} &60.9(-97.0) &125(-99.0) &{\bf *10.5(-99.5)} &{\bf *69.0(-99.5)} \\
&6 &4 &23200 &172000 &{\bf *39.7(-99.8)} &{\bf *302(-99.8)} &233(-99.0) &492(-99.7) &{\bf *39.7(-99.8)} &{\bf *302(-99.8)} \\
\cline{2-11}
&5 &5 &5212 &33413 &{\bf *10.5(-99.8)} &{\bf *69.0(-99.8)} &76.8(-98.5) &158(-99.5) &{\bf *10.5(-99.8)} &{\bf *69.0(-99.8)} \\
&5 &6 &11569 &74220 &{\bf *10.5(-99.9)} &{\bf *69.0(-99.9)} &92.7(-99.2) &191(-99.7) &{\bf *10.5(-99.9)} &{\bf *69.0(-99.9)} \\
&5 &7 &23003 &147653 &{\bf *10.5(-99.9)} &{\bf *69.0(-99.9)} &109(-99.5) &223(-99.8) &{\bf *10.5(-99.9)} &{\bf *69.0(-99.9)} \\
\hline 
\multirow{6}{*}{{\rotatebox[origin=c]{90}{AW}}}
&2 &2 &6.40 &152 &6.40($\pm$0.0) &152($\pm$0.0) &6.40($\pm$0.0) &{\bf *14.7(-90.3)} &6.40($\pm$0.0) &19.1(-87.4) \\
&3 &2 &104 &3267 &104($\pm$0.0) &3267($\pm$0.0) &107(+3.7) &{\bf *359(-89.0)} &104($\pm$0.0) &431(-86.8) \\
&4 &2 &670 &25925 &670($\pm$0.0) &25925($\pm$0.0) &670($\pm$0.0) &{\bf *2813(-89.2)} &670($\pm$0.0) &3108(-88.0) \\
\cline{2-11}
&2 &10 &113 &4734 &{\bf *52.5(-53.5)} &2112(-55.4) &{\bf *52.5(-53.5)} &{\bf *139(-97.1)} &{\bf *52.5(-53.5)} &173(-96.3) \\
&2 &20 &1115 &69578 &{\bf *157(-86.0)} &9470(-86.4) &172(-84.6) &{\bf *523(-99.2)} &{\bf *157(-86.0)} &543(-99.2) \\
&2 &30 &4735 &391001 &{\bf *312(-93.4)} &25170(-93.6) &469(-90.1) &{\bf *469(-99.9)} &{\bf *312(-93.4)} &1109(-99.7) \\
\hline 
\multirow{6}{*}{{\rotatebox[origin=c]{90}{AT}}}
&3 &5 &0.70 &2.55 &{\bf *0.51(-27.3)} &{\bf *1.91(-25.0)} &0.70($\pm$0.0) &2.55($\pm$0.0) &{\bf *0.51(-27.3)} &{\bf *1.91(-25.0)} \\
&4 &5 &6.14 &27.4 &{\bf *3.24(-47.3)} &{\bf *14.7(-46.3)} &6.14($\pm$0.0) &27.4($\pm$0.0) &{\bf *3.24(-47.3)} &{\bf *14.7(-46.3)} \\
&5 &5 &53.2 &276 &{\bf *18.3(-65.6)} &{\bf *94.6(-65.8)} &53.2($\pm$0.0) &276($\pm$0.0) &{\bf *18.3(-65.6)} &{\bf *94.6(-65.8)} \\
\cline{2-11}
&2 &2 &0.04 &0.07 &{\bf *0.03(-14.3)} &{\bf *0.06(-13.7)} &0.04($\pm$0.0) &0.07($\pm$0.0) &{\bf *0.03(-14.3)}&{\bf *0.06(-13.7)} \\
&2 &3 &0.05 &0.11 &{\bf *0.04(-12.5)} &{\bf *0.10(-10.8)} &0.05($\pm$0.0) &0.11($\pm$0.0) &{\bf *0.04(-12.5)} &{\bf *0.10(-10.8)} \\
&2 &4 &0.06 &0.16 &{\bf *0.06(-11.1)} &{\bf *0.14( -8.9)} &0.06($\pm$0.0) &0.16($\pm$0.0) &{\bf *0.06(-11.1)} &{\bf *0.14(-8.9)} \\
\hline 
\multirow{6}{*}{{\rotatebox[origin=c]{90}{BW}}}
&2 &5 &0.14 &0.37 &0.14($\pm$0.0) &0.37($\pm$0.0) &0.14($\pm$0.0) &0.37($\pm$0.0) &0.14($\pm$0.0) &0.37($\pm$0.0) \\
&3 &5 &1.66 &6.40 &1.66($\pm$0.0) &6.40($\pm$0.0) &1.66($\pm$0.0) &6.40($\pm$0.0) &1.66($\pm$0.0) &6.40($\pm$0.0) \\
&4 &5 &19.3 &99.7 &19.3($\pm$0.0) &99.7($\pm$0.0) &19.3($\pm$0.0) &99.7($\pm$0.0) &19.3($\pm$0.0) &99.7($\pm$0.0) \\
\cline{2-11}
&5 &3 &25.8 &158 &25.8($\pm$0.0) &158($\pm$0.0) &25.8($\pm$0.0) &158($\pm$0.0) &25.8($\pm$0.0) &158($\pm$0.0) \\
&5 &4 &85.3 &542 &85.3($\pm$0.0) &542($\pm$0.0) &85.3($\pm$0.0) &542($\pm$0.0) &85.3($\pm$0.0) &542($\pm$0.0) \\
&5 &5 &222 &1446 &222($\pm$0.0) &1446($\pm$0.0) &222($\pm$0.0) &1446($\pm$0.0) &222($\pm$0.0) &1446($\pm$0.0) \\
\hline 
\multirow{5}{*}{{\rotatebox[origin=c]{90}{CM}}}
&2 &2 &3.93 &10.4 &{\bf *2.70(-31.2)} &{\bf *7.09(-31.8)} &3.93($\pm$0.0) &10.4($\pm$0.0) &{\bf *2.70(-31.2)} &{\bf *7.09(-31.8)} \\
&3 &2 &161 &592 &{\bf *66.8(-58.5)} &{\bf *240(-59.5)} &161($\pm$0.0) &592($\pm$0.0) &{\bf *66.8(-58.5)} &{\bf *240(-59.5)} \\
\cline{2-11}
&2 &3 &15.9 &44.9 &{\bf *12.1(-24.3)} &{\bf *33.8(-24.9)} &15.9($\pm$0.0) &44.9($\pm$0.0) &{\bf *12.1(-24.3)} &{\bf *33.8(-24.9)} \\
&2 &4 &45.1 &132 &{\bf *36.3(-19.6)} &{\bf *105(-20.0)} &45.1($\pm$0.0) &132($\pm$0.0) &{\bf *36.3(-19.6)} &{\bf *105(-20.0)} \\
&2 &5 &103 &308 &{\bf *86.3(-16.4)} &{\bf *256(-16.7)} &103($\pm$0.0) &308($\pm$0.0) &{\bf *86.3(-16.4)} &{\bf *256(-16.7)} \\
\hline 
\multirow{6}{*}{{\rotatebox[origin=c]{90}{DC}}}
&2 &2 &4.62 &23.9 &0.78(-83.0) &3.94(-83.6) &0.78(-83.0) &2.86(-88.1) &{\bf *0.50(-89.3)} &{\bf *1.80(-92.5)} \\
&3 &2 &314 &2441 &12.4(-96.1) &88.2(-96.4) &53.3(-83.0) &297(-87.8) &{\bf *9.15(-97.1)} &{\bf *50.2(-97.9)} \\
&4 &2 &21381 &221360 &119(-99.4) &1093(-99.5) &3625(-83.0) &27174(-87.7) &{\bf *98.0(-99.5)} &{\bf *714(-99.7)} \\
\cline{2-11}
&2 &5 &35.3 &416 &16.5(-53.2) &416($\pm$0.0) &{\bf *6.67(-81.1)} &{\bf *160(-61.5)} &10.4(-70.7) &258(-38.0) \\
&2 &6 &52.0 &952 &32.6(-37.2) &952($\pm$0.0) &{\bf *12.9(-75.2)} &{\bf *361(-62.1)} &20.5(-60.6) &592(-37.7) \\
&2 &7 &71.8 &1884 &56.8(-20.9) &1884($\pm$0.0) &{\bf *22.0(-69.3)} &{\bf *705(-62.6)} &35.7(-50.3) &1176(-37.5) \\
\hline 
\multirow{6}{*}{{\rotatebox[origin=c]{90}{AC}}}
&2 &5 &0.53 &1.70 &0.53($\pm$0.0) &1.70($\pm$0.0) &{\bf *0.25(-52.4)} &{\bf *0.72(-57.7)} &{\bf *0.25(-52.4)} &{\bf *0.72(-57.7)} \\
&3 &5 &12.2 &58.7 &12.2($\pm$0.0) &58.7($\pm$0.0) &{\bf *3.53(-71.0)} &{\bf *14.9(-74.7)} &{\bf *3.53(-71.0)} &{\bf *14.9(-74.7)} \\
&4 &5 &280 &1801 &280($\pm$0.0) &1801($\pm$0.0) &{\bf *49.4(-82.3)} &{\bf *275(-84.7)} &{\bf *49.4(-82.3)} &{\bf *275(-84.7)} \\
\cline{2-11}
&5 &2 &161 &1171 &161($\pm$0.0) &1171($\pm$0.0) &{\bf *36.9(-77.1)} &{\bf *233(-80.1)} &{\bf *36.9(-77.1)} &{\bf *233(-80.1)} \\
&5 &3 &759 &5822 &759($\pm$0.0) &5822($\pm$0.0) &{\bf *120(-84.2)} &{\bf *794(-86.4)} &{\bf *120(-84.2)} &{\bf *794(-86.4)} \\
&5 &4 &2476 &19548 &2476($\pm$0.0) &19548($\pm$0.0) &{\bf *311(-87.4)} &{\bf *2115(-89.2)} &{\bf *311(-87.4)} &{\bf *2115(-89.2)} \\
\hline
\hline
\multirow{2}{*}{{\rotatebox[origin=c]{90}{全体}}}
&\multicolumn{2}{c|}{平均} &- &- &-23.4 &-36.4 &-43.2 &-51.7 &-51.8 &-58.3 \\
&\multicolumn{2}{c|}{標準偏差} &- &- &40.0 &41.3 &42.4 &42.4 &38.4 &38.2 \\
\bottomrule
\end{tabular}
\\{\footnotesize ※ $N$:環境モデル数,$K$:環境モデルサイズ,$|S|$:最大状態数(K),$|\Delta|$:最大遷移数(K),「*」付数字は削減効果が最大の結果}
\end{table*}


\begin{table*}[ht]
\centering
\scriptsize
\caption{DCSにおける主記憶量と計算時間(削減率:\%)}
\ecaption{Memory and computation time in DCS (Reduction rate: \%)}
\label{table:cost}
\begin{tabular}{c|cc|rr|rr|rr|rr}
\toprule
\multicolumn{3}{c|}{} &\multicolumn{2}{c|}{basic DCS} &\multicolumn{2}{c|}{CDCS} &\multicolumn{2}{c|}{SDCS} &\multicolumn{2}{c}{CSDCS}\\
\multicolumn{1}{c}{} &\multicolumn{1}{c}{N} &\multicolumn{1}{c|}{K}
&\multicolumn{1}{c}{$M$} &\multicolumn{1}{c|}{$T$}
&\multicolumn{1}{c}{$M$} &\multicolumn{1}{c|}{$T$}
&\multicolumn{1}{c}{$M$} &\multicolumn{1}{c|}{$T$}
&\multicolumn{1}{c}{$M$} &\multicolumn{1}{c}{$T$}\\
\hline 
\multirow{7}{*}{{\rotatebox[origin=c]{90}{HC}}}
&3 &4 &49.2 &1.32 &{\bf *32.9(-33.2)} &{\bf *1.04(-21.7)} &49.6( +0.8) &8.53(+545.9) &{\bf *32.8(-33.4)} &{\bf *1.02(-22.8)} \\
&4 &4 &244 &5.98 &{\bf *67.9(-72.1)} &{\bf *1.68(-72.0)} &259( +6.3) &28.5(+377.1) &{\bf *67.2(-72.4)} &{\bf *1.68(-72.0)} \\
&5 &4 &1908 &22.4 &{\bf *142(-92.6)} &{\bf *2.56(-88.5)} &635(-66.7) &60.8(+171.9) &{\bf *142(-92.5)} &{\bf *2.53(-88.7)} \\
&6 &4 &23984 &1998 &{\bf *515(-97.9)} &{\bf *5.42(-99.7)} &2326(-90.3) &223(-88.9) &{\bf *513(-97.9)} &{\bf *5.53(-99.7)} \\
\cline{2-11}
&5 &5 &5408 &87.5 &{\bf *141(-97.4)} &{\bf *2.56(-97.1)} &759(-86.0) &71.9(-17.8) &{\bf *143(-97.4)} &{\bf *.59(-97.0)} \\
&5 &6 &10483 &410 &{\bf *144(-98.6)} &{\bf *2.50(-99.4)} &938(-91.1) &81.0(-80.2) &{\bf *143(-98.6)} &{\bf *2.60(-99.4)} \\
&5 &7 &20998 &1623 &{\bf *143(-99.3)} &{\bf *2.45(-99.8)} &1094(-94.8) &91.1(-94.4) &{\bf *143(-99.3)} &{\bf *2.58(-99.8)} \\
\hline 
\multirow{6}{*}{{\rotatebox[origin=c]{90}{AW}}}
&2 &2 &257 &5.74 &250( -2.5) &4.34(-24.4) &127(-50.5) &44.2(+669.5) &{\bf *84.5(-67.1)} &{\bf *3.93(-31.5)} \\
&3 &2 &4710 &32.4 &4638( -1.5) &27.7(-14.4) &1372(-70.9) &109(+237.7) &{\bf *955(-79.7)} &{\bf *18.1(-44.2)} \\
&4 &2 &36156 &362 &35986( -0.5) &319( -12.0) &7850(-78.3) &599(+65.4) &{\bf *6388(-82.3)} &{\bf *198(-45.3)} \\
\cline{2-11}
&2 &10 &3246 &31.7 &3018( -7.0) &19.2(-39.6) &1278(-60.6) &240(+657.6) &{\bf *461(-85.8)} &{\bf *9.90(-68.8)} \\
&2 &20 &16756 &261 &13245(-21.0) &73.8(-71.7) &7148(-57.3) &1276(+389.4) &{\bf *1505(-91.0)} &{\bf *23.5(-91.0)} \\
&2 &30 &84530 &2237 &33712(-60.1) &210(-90.6) &22092(-73.9) &6024(+169.3) &{\bf *3292(-96.1)} &{\bf *60.1(-97.3)} \\
\hline 
\multirow{6}{*}{{\rotatebox[origin=c]{90}{AT}}}
&3 &5 &33.1 &1.06 &{\bf *30.4( -8.3)} &{\bf *0.91(-13.8)} &40.0(+20.7) &8.12(+669.7) &{\bf *30.4( -8.3)} &{\bf *0.90(-14.5)} \\
&4 &5 &59.9 &1.73 &{\bf *54.3( -9.2)} &{\bf *1.44(-16.8)} &252(+321.3) &48.1(+2682.0) &{\bf *54.0( -9.8)} &{\bf *1.47(-15.3)} \\
&5 &5 &225 &4.31 &{\bf *198(-11.9)} &{\bf *3.16(-26.6)} &2030(+801.7) &251(+5726.6) &{\bf *201(-10.7)} &{\bf *3.21(-25.5)} \\
\cline{2-11}
&2 &2 &29.0 &0.42 &{\bf *27.4( -5.6)} &{\bf *0.40( -5.3)} &27.3( -5.8) &1.37(+226.3) &{\bf *27.0( -6.9)} &{\bf *0.40( -5.0)} \\
&2 &3 &29.1 &0.51 &{\bf *27.1( -6.8)} &{\bf *0.46(-10.2)} &27.5( -5.5) &1.83(+260.1) &{\bf *27.1( -7.0)} &{\bf *0.47( -7.9)} \\
&2 &4 &29.8 &0.60 &{\bf *27.6( -7.4)} &{\bf *0.53(-12.3)} &28.9( -3.1) &2.43(+302.0) &{\bf *27.6( -9.8)} &{\bf *0.54( -11.3)} \\
\hline 
\multirow{6}{*}{{\rotatebox[origin=c]{90}{BW}}}
&2 &5 &29.7 &0.33 &{\bf *29.2( -1.7)} &{\bf *0.31( -7.6)} &{\bf *29.2( -1.8)} &0.62(+86.7) &{\bf *29.2( -1.7)} &{\bf *0.30( -7.9)} \\
&3 &5 &40.0 &0.67 &{\bf *39.2( -1.9)} &{\bf *0.63( -4.8)} &45.1(+12.8) &1.52(+128.6) &{\bf *39.2( -1.8)} &{\bf *0.63( -5.9)} \\
&4 &5 &221 &2.69 &{\bf *215( -2.5)} &{\bf *2.62( -2.8)} &265(+19.8) &9.34(+247.1) &{\bf *215( -2.6)} &{\bf *2.65( -1.7)} \\
\cline{2-11}
&5 &3 &320 &3.73 &{\bf *313( -2.3)} &{\bf *3.57( -4.2)} &423(+32.1) &12.8(+242.7) &{\bf *312( -2.6)} &{\bf *3.57( -4.3)} \\
&5 &4 &1032 &12.6 &{\bf *1005( -2.7)} &{\bf *12.1( -4.3)} &1369(+32.7) &40.9(+224.1) &{\bf *1004( -2.7)} &{\bf *12.0( -4.6)} \\
&5 &5 &2697 &39.6 &{\bf *2617( -3.0)} &{\bf *37.7( -4.7)} &3627(+34.5) &135(+241.8) &{\bf *2617( -3.0)} &{\bf *37.6( -5.0)} \\
\hline 
\multirow{5}{*}{{\rotatebox[origin=c]{90}{CM}}}
&2 &2 &51.5 &0.77 &{\bf *40.0(-22.4)} &{\bf *0.72( -7.0)} &49.1( -4.8) &3.67(+375.5) &{\bf *40.0(-22.4)} &{\bf *0.72( -7.0)} \\
&3 &2 &521 &5.09 &{\bf *466(-10.6)} &{\bf *3.87(-24.1)} &1199(+130.0) &52.5(+931.5) &{\bf *467(-10.4)} &{\bf *3.78(-25.8)} \\
\cline{2-11}
&2 &3 &99.2 &1.67 &{\bf *93.5( -5.7)} &{\bf *1.49(-10.6)} &138(+39.4) &15.1(+804.1) &{\bf *93.1( -6.1)} &{\bf *1.44(-13.3)} \\
&2 &4 &255 &3.00 &{\bf *237( -6.8)} &{\bf *2.50(-16.0)} &386(+51.5) &42.7(+1324.1) &{\bf *239( -6.2)} &{\bf *2.51(-16.2)} \\
&2 &5 &586 &5.84 &{\bf *543( -7.4)} &{\bf *4.85(-16.9)} &910(+55.3) &104(+1683.3) &{\bf *544( -7.2)} &{\bf *4.85(-17.0)} \\
\hline 
\multirow{6}{*}{{\rotatebox[origin=c]{90}{DC}}}
&2 &2 &35.5 &1.04 &32.3( -9.2) &0.86(-16.9) &34.2( -3.7) &4.11(+296.2) &{\bf *30.5(-14.2)} &{\bf *0.88(-15.6)} \\
&3 &2 &357 &6.99 &168(-52.9) &2.15(-69.3) &666(+86.5) &36.1(+416.5) &{\bf *116(-67.6)} &{\bf *2.20(-68.6)} \\
&4 &2 &25210 &5347 &1680(-93.3) &14.0(-99.7) &52110(+106.7) &13252(+147.9) &{\bf *1135(-95.5)} &{\bf *9.78(-99.8)} \\
\cline{2-11}
&2 &5 &664 &5.51 &650( -2.2) &4.91(-10.9) &651( -2.0) &34.9(+532.3) &{\bf *428(-35.6)} &{\bf *3.95(-28.3)} \\
&2 &6 &1435 &10.4 &1406( -2.0) &9.59( -8.2) &1406( -2.0) &68.9(+559.7) &{\bf *930(-35.2)} &{\bf *7.26(-30.5)} \\
&2 &7 &2793 &18.4 &2762( -1.1) &18.1( -1.9) &2745( -1.7) &124(+575.1) &{\bf *1824(-34.7)} &{\bf *12.7(-31.3)} \\
\hline 
\multirow{6}{*}{{\rotatebox[origin=c]{90}{AC}}}
&2 &5 &30.4 &0.47 &29.8( -1.8) &0.42( -9.4) &{\bf *28.5( -6.3)} &0.92(+97.4) &{\bf *28.3(-6.9)} &{\bf *0.42( -9.9)} \\
&3 &5 &150 &2.09 &145( -3.2) &1.79(-14.4) &{\bf *61.0(-59.3)} &1.84(-11.9) &{\bf *61.6(-58.9)} &{\bf *1.16(-44.3)} \\
&4 &5 &3639 &53.9 &3506( -3.7) &51.0( -5.3) &{\bf *593(-83.7)} &{\bf *7.69(-85.7)} &{\bf *593(-83.7)} &{\bf *6.7(-87.6)} \\
\cline{2-11}
&5 &2 &2247 &26.4 &2156( -4.0) &24.9( -5.8) &{\bf *468(-79.2)} &6.79(-74.3) &{\bf *469(-79.1)} &{\bf *5.41(-79.5)} \\
&5 &2 &10344 &287 &9875( -4.5) &277( -3.6) &{\bf *1577(-84.8)} &{\bf *20.1(-93.0)} &{\bf *1572(-84.8)} &{\bf *18.1(-93.7)} \\
&5 &2 &35025 &2528 &33436( -4.5) &2555( +1.0) &{\bf *4101(-88.3)} &{\bf *69.8(-97.2)} &{\bf *4095(-88.3)} &{\bf *67.6(-97.3)} \\
\hline
\hline
\multirow{2}{*}{{\rotatebox[origin=c]{90}{全体}}}
&\multicolumn{2}{c|}{平均} &- &- &-23.4 &-30.1 &+11.9 &+510.0 &-45.1 &-43.6 \\
&\multicolumn{2}{c|}{標準偏差} &- &- &33.6 &34.2 &146.7 &964.6 &38.2 &36.2 \\
\bottomrule
\end{tabular}
\\{\footnotesize ※ $N$:環境モデル数,$K$:環境モデルサイズ,$M$:必要主記憶量(MB),$T$:計算時間(s),「*」付数字は削減効果が最大の結果}
\end{table*}


\subsection{実験結果と評価}
\label{subsection:result}

\subsubsection{CSDCSはどれほどの削減効果を持つか(RQ1)}
表\ref{table:space}と表\ref{table:cost}より,CSDCSはbasic DCSに対して$|S|$は平均-51.8\%,$|\Delta|$は平均-58.3,$M$は平均-45.1\%,$T$は平均-43.6\%の削減効果が確認された.$|S|$,$|\Delta|$,$M$,$T$全てにおいて,CSDCSは常にbasic DCSと同じかそれ以下となることが確認でき,basic DCSより効率的に制御器を合成できることが確認された.

\subsubsection{CDCSと同等以上の削減効果を持つか(RQ2)}
適用範囲について,CDCSで計算空間削減効果が確認されたHC,AW(一部),AT,CM,DCの全ての場合においてCSDCSでも計算空間削減効果が確認された.
また,CSDCSはCDCSで削減効果が確認されなかったAW(一部),ACにおいても計算空間削減効果が確認された.
以上の結果から,表\ref{table:space}と表\ref{table:cost}においてCSDCSはCDCSの適用範囲を包括しつつも,より広い適用範囲を持つことがわかる.

削減効果量について,CSDCSはCDCSと比較して$|S|$で平均-28.4\%,$|\Delta|$で平均-21.9\%,$M$で平均-21.7\%,$T$で平均-13.5\%の性能向上が確認された.削減効果の標準偏差は,SDCSと比較して$|S|$が-1.6,$|S|$が-3.1だけCSDCSが低い値となっており,CSDCSはCDCSより安定して高い削減効果を$|S|$と$|\Delta|$において発揮できたことが確認された.$M$と$T$の標準偏差は,$M$が+4.5,$T$が+2.1とCSDCSよりもCDCSの方が削減効果が安定していることが確認されたが,CSDCSの方が削減効果の平均値が$M$で-22.0\%,$T$で-13.5\%だけ高い性能を上げており,性能差に比べると標準偏差の差は小さい.
以上の結果から,表\ref{table:space}と表\ref{table:cost}においてCSDCSはCDCSより高い削減効果を同程度安定して発揮できていることがわかる.

以上より,表\ref{table:space}と表\ref{table:cost}において,CSDCSはCDCS以上の適用範囲を持ちつつ,より高い削減効果量を安定して発揮していることがわかった.この結果より,CSDCSはCDCS以上の削減効果を持つと推察される.

\subsubsection{SDCSと同等以上の削減効果を持つか(RQ3)}
適用範囲について,SDCSで計算空間削減効果が確認されたHC,AW,DC,ACの全ての場合においてCSDCSでも計算空間削減効果が確認された.
また,CSDCSはSDCSで削減効果が確認されなかったAT,CMにおいても計算空間削減効果が確認された.
以上の結果から,表\ref{table:space}と表\ref{table:cost}においてCSDCSはSDCSの適用範囲を包括しつつも,より広い適用範囲を持つことがわかる.

削減効果量について,CSDCSはSDCSと比較して$|\Delta|$で平均-4.0\%,$M$で平均-57.0\%,$T$で平均-553.6\%の性能向上が確認された.$|S|$は平均+8.6\%と性能が悪化したが,削減率の標準偏差はSDCSと比較して全てCSDCSが低い値となっており,CSDCSはSDCSより安定して高い削減率を発揮できたことが確認された.特に$T$の削減率は,SDCSでは標準偏差953.3とシステムによって大きくばらついていたが,CSDCSはそのばらつきを標準偏差36.0まで抑制できていることが確認された.
以上の結果から,表\ref{table:space}と表\ref{table:cost}においてCSDCSはSDCSより高い削減効果量をより安定して発揮できていることがわかる.

以上より,表\ref{table:space}と表\ref{table:cost}において,CSDCSはSDCS以上の適用範囲を持ちつつ,より高い削減効果量を安定して発揮していることがわかった.この結果より,CSDCSはSDCS以上の削減効果を持つと推察される.

\subsubsection{適用選択の課題を解消できるか(RQ4)}
本課題を解決する上で,CSDCSはSDCSとCDCSの両手法の適用範囲を包括し,両手法と同等以上の削減効果量を持つ必要がある.

適用範囲について,表\ref{table:space}と表\ref{table:cost}において,CSDCSはSDCSとCDCSのどちらかにおいて計算空間削減効果を持つ場合,常にCSDCSも空間削減効果を持つことがRQ2とRQ3の実験結果よりわかった.
この結果から,CSDCSはSDCSとCDCSの適用範囲を包括すると考えられる.

削減効果量に関して,表\ref{table:space}と表\ref{table:cost}において,CSDCSはSDCSとCDCSのどちらよりも削減効果(削減率)の平均値が高いことがRQ2とRQ3の実験結果よりわかった.
また,CSDCSの標準偏差は,SDCSより低い値,CDCSと同程度であることがRQ2とRQ3の実験結果よりわかった.
この結果から,CSDCSはSDCSとCDCSより高い削減効果量を同程度以上安定して発揮できると考えられる.
以上より,CSDCSは,SDCSとCDCSの適用範囲を包括し,SDCSとCDCSと同程度以上安定して高い削減効果量を持つと推察する.
よってCSDCSは,SDCSとCDCSの適用選択の課題を解決しうると考える.


\subsubsection{議論}
実験結果より,CSDCSはSDCSとCDCSの適用選択の課題を解決しうることが確認された.
加えて,表\ref{table:space}に示すDCの結果から,CSDCSがSDCSおよびCDCSのいずれよりも高い計算空間削減効果を有することも明らかとなった.
この結果は,DCにおいて,SDCSおよびCDCSのそれぞれのアプローチでなければ削減できない状態が存在し,CSDCSが両者の手法を統合することで,それらの状態を同時に削減できたためであると考えられる.
今後は,SDCSとCDCSの統合手法をさらに改良し,削減効果を一層高められる可能性について検討を進める.

また,SDCSとCDCSの両方において削減効果が得られなかった全てのケースでは,CSDCSでも削減効果は見られなかった.
今後は,CSDCSがSDCSやCDCSを超えて,さらに広い適用範囲を持ちうるかどうかについて,追加実験や理論的検証を通じて調査する.

\subsection{妥当性への脅威}
本実験では,$|S|$,$|\Delta|$の計測に関して,basic DCS,SDCS,CDCS,CSDCSにおいて合成アルゴリズム以外の差分がないため,削減効果の内的妥当性を脅かす要因は存在しない.
$M$,$T$の計測に関しては,OSが制御するバックグラウンドプロセスが影響を与えている可能性があるが,その影響はDCSで要求される$M$,$T$と比較して非常に小さな値であるため無視できる.
外的妥当性を脅かす要因としては,実験を行っていない開発対象システムの存在が挙げられる.この外的妥当性への脅威に可能な限り対処するために,性能比較対象であるSDCSとCDCSの評価実験で扱われたシステムにおいてCSDCSの評価実験を行い,恣意的な実験対象の選別を回避した.



% 両立による更なる削減効果について(なぜか)
% SDCSとCDCSと両方において計算空間削減効果を持たない場合,CSDCSが空間削減効果を持つことはなかった.適用範囲については将来研究とする


% {\color{red}
%   [課題1] 削減効果について全体やモデルごとに最大,平均を計算する.また観測された傾向を$N$や$K$を踏まえて示す.

%   [課題2] 表の結果を以下に分類して説明する.
%   \begin{enumerate}[\bf 1.]
%     \item SDCSのみ効果があるもの:Access Management,Auto Warehouse(一部)
%     \item CDCSのみ効果があるもの:AT,CM
%     \item SDCSとCDCSの両方で効果がないもの:BW
%     \item SDCSとCDCSの両方で効果があるもの
%     \begin{enumerate}[\bf i)]
%       \item SDCSとCDCSの両方の効果が重複し,さらに削減したもの:Drone Control, Auto Warehouse(一部)
%       \item CDCSの削減効果が大きくてCDCSと同じ結果になったもの:Headcount Control
%       \item SDCSの削減効果が大きくてSDCSと同じ結果になったもの:(ここにSDCSでAuto warehouse入れれると最高かも)
%     \end{enumerate}
%   \end{enumerate}
%   以上を踏まえた上で,以下結論.
%   \begin{enumerate}[\bf 結論1.]
%       \item CDCSと比較して,削減効果は常に同じかそれ以上
%       \item SDCSと比較して,削減効果はSDCSの方が計算空間は小さくなる場合があったが,計算時間や必要メモリは全ての場合において良くなった
%       \item SDCSとCDCSの両方の効果がないものに対しては削減効果はない
%       \item SDCSとCDCSの両方に削減効果があるシナリオの場合,どちらか削減効果の大きい方が反映されるだけでなく,SDCSとCDCSの両方よりも大きな削減効果を示す場合も観測された
%     \end{enumerate}
%   よって,CSDCSは主記憶量,計算時間の観点からSDCSとCDCSと同等かそれ以上の効果があると思われる.
% }

% 設定されたReserch Questionごとに表\ref{table:space}と表\ref{table:cost}の実験結果を確認し,実験結果を踏まえてReserch Questionの答えを考察する.


% \begin{table*}[ht]
% \centering
% \scriptsize
% \caption{DCSにおける最大の状態数と遷移数(削減率:\%)}
% \ecaption{Maximum number of states and transitions in DCS (Reduction rate: \%)}
% \label{table:space}
% \begin{tabular}{c|cc|rr|rr|rr|rr}
% \toprule
% \multicolumn{3}{c|}{} &\multicolumn{2}{c|}{basic DCS} &\multicolumn{2}{c|}{SDCS} &\multicolumn{2}{c|}{CDCS} &\multicolumn{2}{c}{DCDCS}\\
% \multicolumn{1}{c}{} &\multicolumn{1}{c}{N} &\multicolumn{1}{c|}{K}
% &\multicolumn{1}{c}{$|S|$} &\multicolumn{1}{c|}{$|\Delta|$}
% &\multicolumn{1}{c}{$|S|$} &\multicolumn{1}{c|}{$|\Delta|$}
% &\multicolumn{1}{c}{$|S|$} &\multicolumn{1}{c|}{$|\Delta|$}
% &\multicolumn{1}{c}{$|S|$} &\multicolumn{1}{c}{$|\Delta|$}\\
% \hline 
% \multirow{7}{*}{{\rotatebox[origin=c]{90}{Headcount Control}}}
% &3 &4 &14.1 &61.6 &3.68(-73.9) &6.90(-88.8) &0.66(-95.3) &2.96(-95.2) &0.66(-95.3) &2.96(-95.2) \\
% &4 &4 &277 &1514 &23.1(-91.7) &45.0(-97.0) &4.07(-98.5) &22.5(-98.5) &4.07(-98.5) &22.5(-98.5) \\
% &5 &4 &2016 &12912 &60.9(-97.0) &125(-99.0) &10.5(-99.5) &69.0(-99.5) &10.5(-99.5) &69.0(-99.5) \\
% &6 &4 &23200 &172000 &233(-99.0) &492(-99.7) &39.7(-99.8) &302(-99.8) &39.7(-99.8) &302(-99.8) \\
% \cline{2-11}
% &5 &5 &5212 &33413 &76.8(-98.5) &158(-99.5) &10.5(-99.8) &69.0(-99.8) &10.5(-99.8) &69.0(-99.8) \\
% &5 &6 &11569 &74220 &92.7(-99.2) &191(-99.7) &10.5(-99.9) &69.0(-99.9) &10.5(-99.9) &69.0(-99.9) \\
% &5 &7 &23003 &147653 &109(-99.5) &223(-99.8) &10.5(-99.9) &69.0(-99.9) &10.5(-99.9) &69.0(-99.9) \\
% \hline 
% \multirow{6}{*}{{\rotatebox[origin=c]{90}{Auto Warehouse}}}
% &2 &2 &6.40 &152 &6.40($\pm$0.0) &14.7(-90.3) &6.40($\pm$0.0) &152($\pm$0.0) &6.40($\pm$0.0) &19.1(-87.4) \\
% &3 &2 &104 &3267 &107(+3.7) &359(-89.0) &104($\pm$0.0) &3267($\pm$0.0) &104($\pm$0.0) &431(-86.8) \\
% &4 &2 &670 &25925 &670($\pm$0.0) &2813(-89.2) &670($\pm$0.0) &25925($\pm$0.0) &670($\pm$0.0) &3108(-88.0) \\
% \cline{2-11}
% &2 &10 &113 &4734 &52.5(-53.5) &139(-97.1) &52.5(-53.5) &2112(-55.4) &52.5(-53.5) &173(-96.3) \\
% &2 &20 &1115 &69578 &172(-84.6) &523(-99.2) &157(-86.0) &9470(-86.4) &157(-86.0) &543(-99.2) \\
% &2 &30 &4735 &391001 &469(-90.1) &469(-99.9) &312(-93.4) &25170(-93.6) &312(-93.4) &1109(-99.7) \\
% \hline 
% \multirow{6}{*}{{\rotatebox[origin=c]{90}{Air Traffic}}}
% &3 &5 &0.70 &2.55 &0.70($\pm$0.0) &2.55($\pm$0.0) &0.51(-27.3) &1.91(-25.0) &0.51(-27.3) &1.91(-25.0) \\
% &4 &5 &6.14 &27.4 &6.14($\pm$0.0) &27.4($\pm$0.0) &3.24(-47.3) &14.7(-46.3) &3.24(-47.3) &14.7(-46.3) \\
% &5 &5 &53.2 &276 &53.2($\pm$0.0) &276($\pm$0.0) &18.3(-65.6) &94.6(-65.8) &18.3(-65.6) &94.6(-65.8) \\
% \cline{2-11}
% &2 &2 &0.04 &0.07 &0.04($\pm$0.0) &0.07($\pm$0.0) &0.03(-14.3) &0.06(-13.7) &0.03(-14.3) &0.06(-13.7) \\
% &2 &3 &0.05 &0.11 &0.05($\pm$0.0) &0.11($\pm$0.0) &0.04(-12.5) &0.10(-10.8) &0.04(-12.5) &0.10(-10.8) \\
% &2 &4 &0.06 &0.1g &0.06($\pm$0.0) &0.16($\pm$0.0) &0.06(-11.1) &0.14( -8.9) &0.06(-11.1) &0.14(-8.9) \\
% \hline 
% \multirow{6}{*}{{\rotatebox[origin=c]{90}{Bidding Workflow}}}
% &2 &5 &0.14 &0.37 &0.14($\pm$0.0) &0.37($\pm$0.0) &0.14($\pm$0.0) &0.37($\pm$0.0) &0.14($\pm$0.0) &0.37($\pm$0.0) \\
% &3 &5 &1.66 &6.40 &1.66($\pm$0.0) &6.40($\pm$0.0) &1.66($\pm$0.0) &6.40($\pm$0.0) &1.66($\pm$0.0) &6.40($\pm$0.0) \\
% &4 &5 &19.3 &99.7 &19.3($\pm$0.0) &99.7($\pm$0.0) &19.3($\pm$0.0) &99.7($\pm$0.0) &19.3($\pm$0.0) &99.7($\pm$0.0) \\
% \cline{2-11}
% &5 &3 &25.8 &158 &25.8($\pm$0.0) &158($\pm$0.0) &25.8($\pm$0.0) &158($\pm$0.0) &25.8($\pm$0.0) &158($\pm$0.0) \\
% &5 &4 &85.3 &542 &85.3($\pm$0.0) &542($\pm$0.0) &85.3($\pm$0.0) &542($\pm$0.0) &85.3($\pm$0.0) &542($\pm$0.0) \\
% &5 &5 &222 &1446 &222($\pm$0.0) &1446($\pm$0.0) &222($\pm$0.0) &1446($\pm$0.0) &222($\pm$0.0) &1446($\pm$0.0) \\
% \hline 
% \multirow{5}{*}{{\rotatebox[origin=c]{90}{Cat \& Mouse}}}
% &2 &2 &3.93 &10.4 &3.93($\pm$0.0) &10.4($\pm$0.0) &2.70(-31.2) &7.09(-31.8) &2.70(-31.2) &7.09(-31.8) \\
% &3 &2 &161 &592 &161($\pm$0.0) &592($\pm$0.0) &66.8(-58.5) &240(-59.5) &66.8(-58.5) &240(-59.5) \\
% \cline{2-11}
% &2 &3 &15.9 &44.9 &15.9($\pm$0.0) &44.9($\pm$0.0) &12.1(-24.3) &33.8(-24.9) &12.1(-24.3) &33.8(-24.9) \\
% &2 &4 &45.1 &132 &45.1($\pm$0.0) &132($\pm$0.0) &36.3(-19.6) &105(-20.0) &36.3(-19.6) &105(-20.0) \\
% &2 &5 &103 &308 &103($\pm$0.0) &308($\pm$0.0) &86.3(-16.4) &256(-16.7) &86.3(-16.4) &256(-16.7) \\
% \hline 
% \multirow{6}{*}{{\rotatebox[origin=c]{90}{Drone Control}}}
% &2 &2 &4.62 &23.9 &0.78(-83.0) &2.86(-88.1) &0.78(-83.0) &3.94(-83.6) &0.50(-89.3) &1.80(-92.5) \\
% &3 &2 &314 &2441 &53.3(-83.0) &297(-87.8) &12.4(-96.1) &88.2(-96.4) &9.15(-97.1) &50.2(-97.9) \\
% &4 &2 &21381 &221360 &3625(-83.0) &27174(-87.7) &119(-99.4) &1093(-99.5) &98.0(-99.5) &714(-99.7) \\
% \cline{2-11}
% &2 &5 &35.3 &416 &6.67(-81.1) &160(-61.5) &16.5(-53.2) &416($\pm$0.0) &10.4(-70.7) &258(-38.0) \\
% &2 &6 &52.0 &952 &12.9(-75.2) &361(-62.1) &32.6(-37.2) &952($\pm$0.0) &20.5(-60.6) &592(-37.7) \\
% &2 &7 &71.8 &1884 &22.0(-69.3) &705(-62.6) &56.8(-20.9) &1884($\pm$0.0) &35.7(-50.3) &1176(-37.5) \\
% \hline 
% \multirow{6}{*}{{\rotatebox[origin=c]{90}{Access Control}}}
% &2 &5 &0.53 &1.70 &0.25(-52.4) &0.72(-57.7) &0.53($\pm$0.0) &1.70($\pm$0.0) &0.25(-52.4) &0.72(-57.7) \\
% &3 &5 &12.2 &58.7 &3.53(-71.0) &14.9(-74.7) &12.2($\pm$0.0) &58.7($\pm$0.0) &3.53(-71.0) &14.9(-74.7) \\
% &4 &5 &280 &1801 &49.4(-82.3) &275(-84.7) &280($\pm$0.0) &1801($\pm$0.0) &49.4(-82.3) &275(-84.7) \\
% \cline{2-11}
% &5 &2 &161 &1171 &36.9(-77.1) &233(-80.1) &161($\pm$0.0) &1171($\pm$0.0) &36.9(-77.1) &233(-80.1) \\
% &5 &3 &759 &5822 &120(-84.2) &794(-86.4) &759($\pm$0.0) &5822($\pm$0.0) &120(-84.2) &794(-86.4) \\
% &5 &4 &2476 &19548 &311(-87.4) &2115(-89.2) &2476($\pm$0.0) &19548($\pm$0.0) &311(-87.4) &2115(-89.2) \\
% \hline
% \hline
% \multirow{2}{*}{{\rotatebox[origin=c]{90}{全体}}}
% &\multicolumn{2}{c|}{平均} &- &- &-43.2 &-51.7 &-23.4 &-36.4 &-51.8 &-58.3 \\
% &\multicolumn{2}{c|}{標準偏差} &- &- &42.4 &42.4 &40.0 &41.3 &38.4 &38.2 \\
% \bottomrule
% \end{tabular}
% \\{\footnotesize ※ $N$:環境モデル数,$K$:環境モデルサイズ,$|S|$:最大状態数(K),$|\Delta|$:最大遷移数(K)}
% \end{table*}

% \begin{table*}[ht]
% \centering
% \scriptsize
% \caption{DCSにおける主記憶量と計算時間(削減率:\%)}
% \ecaption{Memory and computation time in DCS (Reduction rate: \%)}
% \label{table:cost}
% \begin{tabular}{c|cc|rr|rr|rr|rr}
% \toprule
% \multicolumn{3}{c|}{} &\multicolumn{2}{c|}{basic DCS} &\multicolumn{2}{c|}{SDCS} &\multicolumn{2}{c|}{CDCS} &\multicolumn{2}{c}{DCDCS}\\
% \multicolumn{1}{c}{} &\multicolumn{1}{c}{N} &\multicolumn{1}{c|}{K}
% &\multicolumn{1}{c}{$M$} &\multicolumn{1}{c|}{$T$}
% &\multicolumn{1}{c}{$M$} &\multicolumn{1}{c|}{$T$}
% &\multicolumn{1}{c}{$M$} &\multicolumn{1}{c|}{$T$}
% &\multicolumn{1}{c}{$M$} &\multicolumn{1}{c}{$T$}\\
% \hline 
% \multirow{7}{*}{{\rotatebox[origin=c]{90}{Headcount Control}}}
% &3 &4 &49.2 &1.32 &49.6( +0.8) &8.53(+545.9) &32.9(-33.2) &1.04(-21.7) &32.8(-33.4) &1.02(-22.8) \\
% &4 &4 &244 &5.98 &259( +6.3) &28.5(+377.1) &67.9(-72.1) &1.68(-72.0) &67.2(-72.4) &1.68(-72.0) \\
% &5 &4 &1908 &22.4 &635(-66.7) &60.8(+171.9) &142(-92.6) &2.56(-88.5) &142(-92.5) &2.53(-88.7) \\
% &6 &4 &23984 &1998 &2326(-90.3) &223(-88.9) &515(-97.9) &5.42(-99.7) &513(-97.9) &5.53(-99.7) \\
% \cline{2-11}
% &5 &5 &5408 &87.5 &759(-86.0) &71.9(-17.8) &141(-97.4) &2.56(-97.1) &143(-97.4) &2.59(-97.0) \\
% &5 &6 &10483 &410 &938(-91.1) &81.0(-80.2) &144(-98.6) &2.50(-99.4) &143(-98.6) &2.60(-99.4) \\
% &5 &7 &20998 &1623 &1094(-94.8) &91.1(-94.4) &143(-99.3) &2.45(-99.8) &143(-99.3) &2.58(-99.8) \\
% \hline 
% \multirow{6}{*}{{\rotatebox[origin=c]{90}{Auto Warehouse}}}
% &2 &2 &257 &5.74 &127(-50.5) &44.2(+669.5) &250( -2.5) &4.34(-24.4) &84.5(-67.1) &3.93(-31.5) \\
% &3 &2 &4710 &32.4 &1372(-70.9) &109(+237.7) &4638( -1.5) &27.7(-14.4) &955(-79.7) &18.1(-44.2) \\
% &4 &2 &36156 &362 &7850(-78.3) &599(+65.4) &35986( -0.5) &319( -12.0) &6388(-82.3) &198(-45.3) \\
% \cline{2-11}
% &2 &10 &3246 &31.7 &1278(-60.6) &240(+657.6) &3018( -7.0) &19.2(-39.6) &461(-85.8) &9.90(-68.8) \\
% &2 &20 &16756 &261 &7148(-57.3) &1276(+389.4) &13245(-21.0) &73.8(-71.7) &1505(-91.0) &23.5(-91.0) \\
% &2 &30 &84530 &2237 &22092(-73.9) &6024(+169.3) &33712(-60.1) &210(-90.6) &3292(-96.1) &60.1(-97.3) \\
% \hline 
% \multirow{6}{*}{{\rotatebox[origin=c]{90}{Air Traffic}}}
% &3 &5 &33.1 &1.06 &40.0(+20.7) &8.12(+669.7) &30.4( -8.3) &0.91(-13.8) &30.4( -8.3) &0.90(-14.5) \\
% &4 &5 &59.9 &1.73 &252(+321.3) &48.1(+2682.0) &54.3( -9.2) &1.44(-16.8) &54.0( -9.8) &1.47(-15.3) \\
% &5 &5 &225 &4.31 &2030(+801.7) &251(+5726.6) &198(-11.9) &3.16(-26.6) &201(-10.7) &3.21(-25.5) \\
% \cline{2-11}
% &2 &2 &29.0 &0.42 &27.3( -5.8) &1.37(+226.3) &27.4( -5.6) &0.40( -5.3) &27.0( -6.9) &0.40( -5.0) \\
% &2 &3 &29.1 &0.51 &27.5( -5.5) &1.83(+260.1) &27.1( -6.8) &0.46(-10.2) &27.1( -7.0) &0.47( -7.9) \\
% &2 &4 &29.8 &0.60 &28.9( -3.1) &2.43(+302.0) &27.6( -7.4) &0.53(-12.3) &27.6( -9.8) &0.54( -11.3) \\
% \hline 
% \multirow{6}{*}{{\rotatebox[origin=c]{90}{Bidding Workflow}}}
% &2 &5 &29.7 &0.33 &29.2( -1.8) &0.62(+86.7) &29.2( -1.7) &0.31( -7.6) &29.2( -1.7) &0.30( -7.9) \\
% &3 &5 &40.0 &0.67 &45.1(+12.8) &1.52(+128.6) &39.2( -1.9) &0.63( -4.8) &39.2( -1.8) &0.63( -5.9) \\
% &4 &5 &221 &2.69 &265(+19.8) &9.34(+247.1) &215( -2.5) &2.62( -2.8) &215( -2.6) &2.65( -1.7) \\
% \cline{2-11}
% &5 &3 &320 &3.73 &423(+32.1) &12.8(+242.7) &313( -2.3) &3.57( -4.2) &312( -2.6) &3.57( -4.3) \\
% &5 &4 &1032 &12.6 &1369(+32.7) &40.9(+224.1) &1005( -2.7) &12.1( -4.3) &1004( -2.7) &12.0( -4.6) \\
% &5 &5 &2697 &39.6 &3627(+34.5) &135(+241.8) &2617( -3.0) &37.7( -4.7) &2617( -3.0) &37.6( -5.0) \\
% \hline 
% \multirow{5}{*}{{\rotatebox[origin=c]{90}{Cat \& Mouse}}}
% &2 &2 &51.5 &0.77 &49.1( -4.8) &3.67(+375.5) &40.0(-22.4) &0.72( -7.0) &40.0(-22.4) &0.72( -7.0) \\
% &3 &2 &521 &5.09 &1199(+130.0) &52.5(+931.5) &466(-10.6) &3.87(-24.1) &467(-10.4) &3.78(-25.8) \\
% \cline{2-11}
% &2 &3 &99.2 &1.67 &138(+39.4) &15.1(+804.1) &93.5( -5.7) &1.49(-10.6) &93.1( -6.1) &1.44(-13.3) \\
% &2 &4 &255 &3.00 &386(+51.5) &42.7(+1324.1) &237( -6.8) &2.50(-16.0) &239( -6.2) &2.51(-16.2) \\
% &2 &5 &586 &5.84 &910(+55.3) &104(+1683.3) &543( -7.4) &4.85(-16.9) &544( -7.2) &4.85(-17.0) \\
% \hline 
% \multirow{6}{*}{{\rotatebox[origin=c]{90}{Drone Control}}}
% &2 &2 &35.5 &1.04 &34.2( -3.7) &4.11(+296.2) &32.3( -9.2) &0.86(-16.9) &30.5(-14.2) &0.88(-15.6) \\
% &3 &2 &357 &6.99 &666(+86.5) &36.1(+416.5) &168(-52.9) &2.15(-69.3) &116(-67.6) &2.20(-68.6) \\
% &4 &2 &25210 &5347 &52110(+106.7) &13252(+147.9) &1680(-93.3) &14.0(-99.7) &1135(-95.5) &9.78(-99.8) \\
% \cline{2-11}
% &2 &5 &664 &5.51 &651( -2.0) &34.9(+532.3) &650( -2.2) &4.91(-10.9) &428(-35.6) &3.95(-28.3) \\
% &2 &6 &1435 &10.4 &1406( -2.0) &68.9(+559.7) &1406( -2.0) &9.59( -8.2) &930(-35.2) &7.26(-30.5) \\
% &2 &7 &2793 &18.4 &2745( -1.7) &124(+575.1) &2762( -1.1) &18.1( -1.9) &1824(-34.7) &12.7(-31.3) \\
% \hline 
% \multirow{6}{*}{{\rotatebox[origin=c]{90}{Access Control}}}
% &2 &5 &30.4 &0.47 &28.5( -6.3) &0.92(+97.4) &29.8( -1.8) &0.42( -9.4) &28.3(-6.9) &0.42( -9.9) \\
% &3 &5 &150 &2.09 &61.0(-59.3) &1.84(-11.9) &145( -3.2) &1.79(-14.4) &61.6(-58.9) &1.16(-44.3) \\
% &4 &5 &3639 &53.9 &593(-83.7) &7.69(-85.7) &3506( -3.7) &51.0( -5.3) &593(-83.7) &6.7(-87.6) \\
% \cline{2-11}
% &5 &2 &2247 &26.4 &468(-79.2) &6.79(-74.3) &2156( -4.0) &24.9( -5.8) &469(-79.1) &5.41(-79.5) \\
% &5 &2 &10344 &287 &1577(-84.8) &20.1(-93.0) &9875( -4.5) &277( -3.6) &1572(-84.8) &18.1(-93.7) \\
% &5 &2 &35025 &2528 &4101(-88.3) &69.8(-97.2) &33436( -4.5) &2555( +1.0) &4095(-88.3) &67.6(-97.3) \\
% \hline
% \hline
% \multirow{2}{*}{{\rotatebox[origin=c]{90}{全体}}}
% &\multicolumn{2}{c|}{平均} &- &- &+11.9 &+510.0 &-23.4 &-30.1 &-45.1 &-43.6 \\
% &\multicolumn{2}{c|}{標準偏差} &- &- &146.7 &964.6 &33.6 &34.2 &38.2 &36.2 \\
% \bottomrule
% \end{tabular}
% \\{\footnotesize ※ $N$:環境モデル数,$K$:環境モデルサイズ,$M$:必要主記憶量(MB),$T$:計算時間(s)}
% \end{table*}

\section{関連研究}
\label{section:relatedwork}
CSDCSのように,検証時の計算空間爆発を抑制するために,システム全体の検証問題をシステムの構成要素ごとのサブ問題に分解・検証するアプローチはCompositional Verificationと呼ばれ,モデル検査分野で広く研究されてきた\cite{paper:CompositionalVerification_1}\cite{paper:CompositionalVerification_2}\cite{paper:CompositionalVerification_3}\cite{paper:CompositionalVerification_4}\cite{paper:CompositionalVerification_5}\cite{paper:CompositionalVerification_6}\cite{paper:CompositionalVerification_7}\cite{paper:CompositionalVerification_8}.
しかし,安全性を充足可能なシステム全体の状態空間を制御器として出力する必要があるDCSでは,サブ問題の解がシステム全体の問題の状態空間のどの状態に該当するか特定する必要がある.よって,問題をサブ問題として解くだけではDCSの計算空間削減は実現できず,DCS分野にはCompositional Verificationは不適だとされてきた.

そこでCSDCSは,Compositional Verificationによって生じるサブ問題(部分合成)の直接的な解である充足可能領域ではなく,Safety Game Solvingで安全性を違反する状態$S_{err}$となった非充足可能領域に着目した.
部分合成で判明した非充足可能領域を,以降の全ての合成プロセスにおいて構築回避するよう,DCSプロセスから見直すことで,DCS分野においてもCompositional Verificationが計算空間削減に有効であることを示した.

CSDCS以外にも,DCS分野では計算空間爆発に対処するアプローチとして,On-the-fly SynthesisやAbstraction Synthesisが研究されてきた.

On-the-fly Synthesisを用いたDCS\cite{paper:On-the-flySyntehsis_1}\cite{paper:On-the-flySyntehsis_2}\cite{paper:On-the-flySyntehsis_3}\cite{paper:On-the-flySyntehsis_4}では,環境モデルと監視モデルに加えて目標状態が与えられ,安全性を違反することなく初期状態から目標状態に到達可能な状態遷移列を制御器として合成する.安全性を違反することのない状態空間を初期状態から逐次探索かつ構築し,目標状態に到達次第探索を終了する.これによりシステムの全状態空間の構築を回避し,計算空間削減を実現した.
しかし,初期状態から目標状態までの制御器しか導出できないため,On-the-fly Synthesisでは目標状態のない継続的な稼動を目的とするシステムの制御器の合成はできない限界が存在する.
対してCSDCSは,安全性を充足するシステムの全状態空間を制御器として導出できることから,このような限界は存在しない.

Abstraction Synthesisを用いたDCS\cite{paper:SynthesisAbstraction_1}\cite{paper:SynthesisAbstraction_2}\cite{aizawa:IEICEJ2020}では,安全性充足を分析する上で詳細である必要のない状態空間を双模倣性を維持しつつ抽象化することで,構築されるゲーム空間の状態数を削減した.しかし,CSDCSと異なり安全性と関わらない一部の状態遷移が抽象化された制御器が合成されるため,システム全体の詳細な制御を定める必要のある開発初期段階における動作仕様設計には用いることができない限界が存在する.

以上,CSDCS以外のDCS計算空間削減手法の多くは,制御器の用途を制限することで,その用途で不要となる計算空間の構築を回避してきた.
対してCSDCSは,従来と同じ制御器を合成しつつ状態削減可能であるため,制御器の用途が制限されない.
そのため,制御器の用途を制限することで,CSDCSは他のDCS計算空間削減手法との両立も可能と考えられるため,用途に応じた更なる計算空間削減手法の拡張が期待される.
\section{あとがき}
\label{section:conclusion}
本論文では,CDCSとSDCSにおける適用選択の課題を解消するために,両手法と比較して同等以上の削減可能対象と状態空間削減効果を備えるCSDCSを提案した.
SDCSの段階的なゲーム空間構築アプローチを基盤としつつ,ゲーム空間構築処理にConsolidated Game Compositionを採用することで,トレードオフの関係にあった高い計算空間削減効果と短い計算時間の両立を実現した.
このCSDCSによって,SDCSとCDCSの使い分けに関する判断工程を不要とし,両者の適用選択における課題を解消した.
今後の研究課題としては,CSDCS アルゴリズムを制御器の用途に応じて最適化し,さらなる計算空間の削減を図ることが挙げられる.


\bibliographystyle{unsrt}
\bibliography{*参考文献}

\begin{biography}
\profile{n}{山内 拓人}{2021年に早稲田大学理工学術院基幹理工学研究科情報理工・情報通信専攻修士課程修了.2021年より早稲田大学理工学術院基幹理工学研究科情報理工・情報通信専攻博士後期課程に所属.早稲田大学助手.自己適応システム,離散制御器合成の研究に従事.}%
\profile{n}{李 家隆}{2021年に早稲田大学理工学術院基幹理工学研究科情報理工・情報通信専攻修士課程修了.2021年より早稲田大学理工学術院基幹理工学研究科情報理工・情報通信専攻博士後期課程に所属,現在に至る.自己適応システム,要求工学の研究に従事.}
\profile{m,E}{鄭 顕志}{2008 年早稲田大学大学院理工学研究科博士課程修了.同大学助手,助教,国立情報学研究所助教,准教授,早稲田大学研究院准教授/主任研究員を経て 2019 年より早稲田大学准教授.現在に至る.現在,国立情報学研究所 GRACEセンター特任准教授を兼任.博士 (工学) (早稲田大学).自己適応システム,ソフトウェアアーキテクチャ,モデル駆動工学の研究に従事.}
% \profile{m,E}{鷲崎 弘宜}{早稲田大学 研究推進部 副部長,早稲田大学グローバルソフトウェアエンジニアリング研究所所長,早稲田大学理工学術院基幹理工学部情報理工学科教授,国立情報学研究所客員教授,株式会社システム情報取締役(監査等委員),株式会社エクスモーション 社外取締役,ガイオ・テクノロジー株式会社 技術アドバイザ.IEEE Computer Society PEAB Engineering Discipline Committee Chair,ISO/IEC/JTC1 SC7/WG20 Convenor,enPiT-Pro スマートエスイー事業責任者.ビジネスと社会のためのスマートなソフトウェアエンジニアリングの研究,教育,社会実装に従事.}
\profile{h,L}{本位田 真一}{1978年早稲田大学大学院理工学研究科修士課程修了.(株)東芝を経て2000年より国立情報学研究所教授,2012年より同研究所副所長を兼務.2001年より東京大学大学院情報理工学系研究科教授を兼任,2018年より早稲田大学理工学術院教授,現在に至る.現在,英国UCL客員教授を兼任.2005年度パリ第6大学招聘教授.2015年度リヨン第1大学招聘教授.日本学術会議連携会員.情報処理学会フェロー.}
\end{biography}



\end{document}
