\section{関連研究}
\label{section:relatedwork}
CSDCSのように,検証時の計算空間爆発を抑制するために,システム全体の検証問題をシステムの構成要素ごとのサブ問題に分解・検証するアプローチはCompositional Verificationと呼ばれ,モデル検査分野で広く研究されてきた\cite{paper:CompositionalVerification_1}\cite{paper:CompositionalVerification_2}\cite{paper:CompositionalVerification_3}\cite{paper:CompositionalVerification_4}\cite{paper:CompositionalVerification_5}\cite{paper:CompositionalVerification_6}\cite{paper:CompositionalVerification_7}\cite{paper:CompositionalVerification_8}.
しかし,安全性を充足可能なシステム全体の状態空間を制御器として出力する必要があるDCSでは,サブ問題の解がシステム全体の問題の状態空間のどの状態に該当するか特定する必要がある.よって,問題をサブ問題として解くだけではDCSの計算空間削減は実現できず,DCS分野にはCompositional Verificationは不適だとされてきた.

CSDCSは,Compositional Verificationによって生じるサブ問題(部分合成)の直接的な解である充足可能領域ではなく,違反状態となった非充足可能領域に着目した.
サブ問題で判明した非充足可能領域を,以降の全てのDCSプロセスにおいて構築回避するよう,DCSプロセスから見直すことで,DCS分野においてもCompositional Verificationが計算空間削減に有効であることを示した.よって,CSDCSはDCS分野におけるCompositional Verification適用のさきがけ的研究であると思われる.

Abstraction Synthesis以外にも,DCS分野では計算空間爆発に対処するアプローチとして,On-the-fly SynthesisやAbstraction Synthesisが研究されてきた.

On-the-fly Synthesisを用いたDCS\cite{paper:On-the-flySyntehsis_1}\cite{paper:On-the-flySyntehsis_2}\cite{paper:On-the-flySyntehsis_3}\cite{paper:On-the-flySyntehsis_4}では,環境モデルと監視モデルに加えて目標状態が与えられ,安全性を違反することなく初期状態から目標状態に到達可能な状態遷移列を制御器として合成する.安全性を違反することのない状態空間を初期状態から逐次探索かつ構築し,目標状態に到達次第探索を終了する.これによりシステムの全状態空間の構築を回避し,計算空間削減を実現した.
しかし,初期状態から目標状態までの制御器しか導出できないため,On-the-fly Synthesisでは目標状態のない継続的な稼動を目的とするシステムの制御器の合成はできない限界が存在する.
対してCSDCSは,安全性を充足するシステムの全状態空間を制御器として導出できることから,このような限界は存在しない.

Abstraction Synthesisを用いたDCS\cite{paper:SynthesisAbstraction_1}\cite{paper:SynthesisAbstraction_2}\cite{aizawa:IEICEJ2020}では,安全性充足を分析する上で詳細である必要のない状態空間を双模倣性を維持しつつ抽象化することで,構築されるゲーム空間の状態数を削減した.しかし,CSDCSと異なり安全性と関わらない一部の状態遷移が抽象化された制御器が合成されるため,システム全体の詳細な制御を定める必要のある開発初期段階における動作仕様設計には用いることができない限界が存在する.

以上,CSDCS以外のDCS計算空間削減手法の多くは,制御器の用途を制限することで,その用途で不要となる計算空間の構築を回避してきた.
対してCSDCSは,従来と同じ制御器を合成しつつ状態削減可能であるため,制御器の用途が制限されない.
そのため,制御器の用途を制限することで,CSDCSは他のDCS計算空間削減手法との両立も可能と考えられるため,用途に応じた更なる計算空間削減手法の拡張が期待される.