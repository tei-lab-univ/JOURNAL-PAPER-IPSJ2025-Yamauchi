\section{まえがき}
\label{section:introduction}
事象の発生によりシステムの状態が離散的に遷移する離散事象システムの開発では,開発の早期段階において安全性が保証された動作仕様を定めることが重要となる.
想定される動作環境下において安全性が保証された動作仕様を自動合成する技術のひとつとして,Discrete Controller Synthesis(DCS)\cite{paper:SupervisoryControl}に関する研究が進められてきた.

DCSを用いた動作仕様設計は次の手順で行われる.はじめに,開発者はシステムの動作環境を仮定し,その特性を環境モデルとしてモデル化する.次に,環境モデル下で保証すべき安全性を定義し,その安全性の充足状況を監視する監視モデルを作成する.最後に,環境モデルと監視モデルを入力としてDCSを実施し,環境下で安全性が保証された動作仕様を表す制御器を自動合成する.DCSは,モデル検査やソフトウェアテストにおいて繰り返し手動で行われていた動作仕様策定と検証を自動化することで,開発者の負担を軽減する.

しかしながら,DCSには計算空間が指数関数的に増加する課題があり,実践的な規模のシステム開発への適用を困難にしている.
DCSは,制御器を合成する過程で環境モデルと監視モデルから安全性を満たす状態空間を分析するためのゲーム空間を構築する.
このゲーム空間は,事象の同期を考慮した環境モデルと監視モデルの全状態の直積により状態を構築するため,それぞれのモデル数の増加に伴って状態数が指数関数的に増加する.
このゲーム空間の指数関数的増加に伴って,要求される計算空間,計算時間,必要主記憶量も指数関数的に増加するため,DCSにおいて計算空間の状態削減は重要な課題となっている.

Consolidated Discrete Controller Synthesis(CDCS)\cite{yamauchi:IPSJ2024}とStepwise Discrete Controller Synthesis(SDCS)\cite{yamauchi:IEICEJ2023}はDCSのゲーム空間構築における状態空間の削減に取り組み,DCSにおける計算空間の指数爆発を抑制した.
CDCS\cite{yamauchi:IPSJ2024}では,ゲーム空間構築過程で安全性を違反すると判明した状態を抽象化しつつゲーム空間を構築する手法,Consolidated Game Compositionを提案した.安全性ゲームを解くにあたってゲーム空間構築過程で安全性を違反すると判明した状態は探索する必要がない.
よって,Consolidated Game Compositionを用いてゲーム空間を構築することで,ゲーム空間構築過程で安全性を違反すると判明した状態の構築を回避し,計算空間の状態削減を実現した.
SDCS\cite{yamauchi:IEICEJ2023}では,監視モデルごとに必要最小限の構成でゲーム空間の構築と分析を行う部分合成を提案した.
そして,部分合成によって安全性を違反すると判明した状態を,他の監視モデルの安全性ゲームにおいて構築回避しつつ,段階的にゲーム空間を構築することで計算空間の状態削減を実現した.

これらCDCSとSDCSは,どちらも安全性を違反する状態に着目したDCS計算空間削減手法であるが,どちらの手法が有効であるかは扱う環境モデルと監視モデルの組み合わせごとに異なる.
そのため,開発者は開発対象に応じてCDCSとSDCSを適切に使い分ける必要があるが,判断するには状態爆発を引き起こす中間生成モデルの構造をすべて把握する必要があり,どちらが有効であるか判断することは現実的に困難である.

そこで本論文では,CDCSとSDCSにおける適用選択の課題を解消するために,SDCSとCDCSを一元化するDCS,Consolidated Stepwise Discrete Controller Synthesis(CSDCS)を提案する.
一元化にあたって,CSDCSは,SDCSおよびCDCSのいずれと比較しても同等以上の削減可能対象と状態空間削減効果を備える必要がある.
そこで,SCDCSでは,SDCSの部分合成を用いた段階的なゲーム空間構築アプローチを基盤としつつ,部分合成のゲーム空間構築処理にConsolidated Game Compositionを導入することで,CDCSとSDCSの両手法の削減可能対象の包含を実現する.
また,Consolidated Game Compositionの採用によって,高い計算空間削減効果を維持したまま,CSDCSにおける部分合成の回数を抑制することが可能となり,SDCSとCDCSの両方と比較しても短い時間で同等以上の削減効果を発揮することが可能になった.
このCSDCSによって,従来必要であったSDCSとCDCSの使い分けに関する判断工程が不要となり,両者の適用選択における課題を解消する.

本論文の貢献は以下の通りである.
\begin{itemize}
	\item CSDCSの合成アルゴリズムの定式化
	\item CSDCSが合成する制御器の妥当性証明
	\item CSDCSの計算空間削減効果の性能評価
\end{itemize}
CSDCSの性能評価では,SDCSやCDCSの評価で用いられた7つの離散事象システムにおいて制御器の合成実験を実施した.実験の結果,CSDCSは平均43.6\%の計算時間を削減しつつも,平均51.8\%計算空間削減効果を持つことがわかり,SDCSとCDCSの両方と同程度以上の削減効果を持つことが確認された.結果,SDCSとCDCSの適用選択の課題を解消しうることがわかった.

本論文の構成は以下の通りである.
\ref{section:DCS}章で背景技術であるDCSについて詳説する.
続く\ref{section:advDCS}章ではSDCSとCDCSがどのようにDCSの計算空間を削減したかについて詳説し,SDCSとCDCSにおける適用選択の課題について示す.
その後,適用選択の課題を解消するCSDCSの仕組みについて\ref{section:proposal}章で詳説し,CSDCSがSDCSとCDCSにおける適用選択の課題を解消しうるかについて\ref{section:evaluation}章で評価実験を通して評価する.
最後に,\ref{section:relatedwork}章で関連研究の説明の後,\ref{section:conclusion}章で結論を述べる.