\section{計算空間の指数爆発抑制に取り組んだDCS}
\label{section:advDCS}
本章では,basicDCSにおける計算空間の指数爆発の抑制に取り組んだDCS,CDCSとSDCSについて詳説する.

% ---------------------------------------------------------------------------------------------------------- %

\subsection{CDCS}
\label{subsection:CDCS}
CDCS\cite{yamauchi:IPSJ2024}は,監視モデルの$s_{err}$に該当する全ての状態を,ひとつの$s_{err}$に抽象化しつつ,ゲーム空間構築することで,DCSの計算空間を削減する.
安全性ゲームを解く上で,ゲーム空間の$s_{err}$から伸びる遷移は探索する必要がない.
そこで,ゲーム空間構築過程で複雑な$s_{err}$の構築を回避することで,計算空間削減を実現する.
CDCSのアルゴリズムをAlgorithm\ref{algorithm:CDCS}に示す.

\begin{algorithm}[h]
\caption{CDCS}
\label{algorithm:CDCS}
\begin{algorithmic}[1]
\renewcommand{\algorithmicrequire}{\textbf{Input:}}
\renewcommand{\algorithmicensure}{\textbf{Output:}}
\REQUIRE $E$ {\bf such that} $\forall e \in E, e = (id_{e}, S_{e}, A_{e}, \Delta_{e}, s^0_{e})$
\REQUIRE $R$ {\bf such that} $\forall r \in R, r = (id_{r}, S_{r}, A_{r}, \Delta_{r}, s^0_{r})$
\ENSURE  $c$

\STATE $g^* =$ {\bf Consolidated Game Composition($E \cup R$})
\STATE $c   =$ {\bf Safety Game Solving($g^*$)}
\STATE return $c$
\end{algorithmic}
\end{algorithm}

Algorithm\ref{algorithm:CDCS}では,Algorithm\ref{algorithm:basic_controller_synthesis}におけるParallel Composition,Modified Parallel Composition,Error State Abstraction全てを一括で行うConsolidated Game Compositionを用いて,$E$と$R$から直接$g^*$を構築する(1行目).その後,basic DCSと同様に$g^*$を入力としてSafety Game Solvingを用いて$c$を導出する.
CDCSを実現するConsolidated Game Compositionは定義\ref{def:consolidated_game_composition}で定義される.

\begin{dfn}{\textbf{Consolidated Game Composition}}
\label{def:consolidated_game_composition}
    記号$\between$で表され,環境モデル群$E$と監視モデル群$R$に含まれる全てのLTSの集合 $E \cup R$を入力とし,$g^*$を直接構築し,出力する.
    LTS $x,y \in E \cup R$が入力される時,$x = (S_{x}, A_{x}, \Delta_{x}, s^0_{x})$,
    $y = (S_{y}, A_{y}, \Delta_{y}, s^0_{y})$とする.
    $g^* = x \between y = (S, A, \Delta, s^0)$の時,初期状態$s^0$は状態の集合$\{ s^0_{x}, s^0_{y} \}$で表され,遷移$(s,a,s')\in\Delta$は式\ref{formula:pc1},\ref{formula:pc2},\ref{formula:pc3}を満たす初期状態$s^0$から到達可能な全状態間を結ぶ全遷移の集合である.
    \begin{multline}
    \label{formula:pc1}
    \Delta_x \langle s_{x} \overset{a}{\rightarrow} s'_{x} \rangle, \Delta_y \langle s_{y} \overset{a}{\nrightarrow} s'_{y} \rangle, a \in A_{x} \backslash A_{y}, s_{err} \notin s'_{x}\\
    \Rightarrow (\{ s_{x},s_{y} \},a,\{ s'_{x},s_{y} \} ) \in \Delta
    \end{multline}
    \begin{multline}
    \label{formula:pc2}
    \Delta_x \langle s_{x} \overset{a}{\nrightarrow} s'_{x} \rangle, \Delta_y \langle s_{y} \overset{a}{\rightarrow} s'_{y} \rangle, a \in A_{y} \backslash A_{x}, s_{err} \notin s'_{y}\\
    \Rightarrow (\{ s_{x},s_{y} \},a,\{ s_{x},s'_{y} \}) \in \Delta
    \end{multline}
    \begin{multline}
    \label{formula:pc3}
    \Delta_x \langle s_{x} \overset{a}{\rightarrow} s'_{x} \rangle, \Delta_y \langle s_{y} \overset{a}{\rightarrow} s'_{y} \rangle, a \in A_{x} \cap A_{y}, s_{err} \notin s'_{x} \cup s'_{y}\\
    \Rightarrow (\{ s_{x},s_{y} \},a,\{ s'_{x},s'_{y} \}) \in \Delta
    \end{multline}
    \begin{multline}
    \label{formula:pc4}
    \Delta_x \langle s_{x} \overset{a}{\rightarrow} s'_{x} \rangle, \Delta_y \langle s_{y} \overset{a}{\nrightarrow} s'_{y} \rangle, a \in A_{x} \backslash A_{y}, s_{err} \in s'_{x}\\
    \Rightarrow (\{ s_{x},s_{y} \},a,s_{err}) \in \Delta
    \end{multline}
    \begin{multline}
    \label{formula:pc5}
    \Delta_x \langle s_{x} \overset{a}{\nrightarrow} s'_{x} \rangle, \Delta_y \langle s_{y} \overset{a}{\rightarrow} s'_{y} \rangle, a \in A_{y} \backslash A_{x}, s_{err} \in s'_{y}\\
    \Rightarrow (\{ s_{x},s_{y} \},a,s_{err}) \in \Delta
    \end{multline}
    \begin{multline}
    \label{formula:pc6}
    \Delta_x \langle s_{x} \overset{a}{\rightarrow} s'_{x} \rangle, \Delta_y \langle s_{y} \overset{a}{\rightarrow} s'_{y} \rangle, a \in A_{x} \cap A_{y}, s_{err} \in s'_{x} \cup s'_{y}\\
    \Rightarrow (\{ s_{x},s_{y} \},a,s_{err}) \in \Delta
    \end{multline}
    % $\Delta_x \langle s_{x} \overset{a}{\rightarrow} s'_{x} \rangle$は$s_{x}$において$a$が生じた場合,状態が$s'_{x}$となる遷移が$\Delta_x$に含まれることを表し,$\Delta_x \langle s_{x} \overset{a}{\nrightarrow} s'_{x} \rangle$は$s_{x}$において$a$が生じた場合,状態が$s'_{x}$となる遷移が$\Delta_x$に含まれないことを表す.
    $S$は$\Delta$に含まれる全状態の集合,$A$は$\Delta$に含まれる全事象の集合で表され,$\Delta$が合成されたとき一意に定まる.
    以上の$E$と$R$から$g^*$を直接導出する作業を"{\bf Consolidated Game Composition ($E \cup R$)}"で表す.
\end{dfn}

このConsolidated Game Compositionを用いることによって,$s_{err}$が重ね合わされる全状態をひとつの$s_{err}$として圧縮しつつ,$g^*$を構築できる.これにより,$s_{err}$が重ね合わされる状態の数だけ計算空間が削減される.


% ---------------------------------------------------------------------------------------------------------- %

\subsection{SDCS}
\label{subsection:SDCS}
SDCS\cite{yamauchi:IEICEJ2023}は,部分合成を段階的に行うことで,DCSの計算空間を削減する.
部分合成とは,システムの部分的な制御器を合成する工程を意味し,システムを構成する一部の環境モデルと監視モデルにParallel Composition,Modified Parallel Composition,Error State Abstraction,Safety Game Solvingを適用することで実現される.
部分制御器を用いてゲーム空間の構築と分析を段階的に行うことで,先の分析で安全性非充足となった状態空間の構築を以降全ての分析で回避できるため,計算空間削減を実現する.
SDCSのアルゴリズムをAlgorithm\ref{algorithm:SDCS}に示す.

\begin{algorithm}[h]
\caption{SDCS}
\label{algorithm:SDCS}
\begin{algorithmic}[1]
\renewcommand{\algorithmicrequire}{\textbf{Input:}}
\renewcommand{\algorithmicensure}{\textbf{Output:}}
\REQUIRE $E$ {\bf such that} $\forall e \in E, e = (id_{e}, S_{e}, A_{e}, \Delta_{e}, s^0_{e})$
\REQUIRE $R$ {\bf such that} $\forall r \in R, r = (id_{r}, S_{r}, A_{r}, \Delta_{r}, s^0_{r})$
\ENSURE $c$
\STATE $\Lambda =$ {\bf Algorithm \ref{algorithm:huristic}($E$, $R$, $1$)}
\FOR {$i^* = 1$ to $n(\Lambda)$}
    \STATE $\lambda = (i^*, eids_{\lambda}, rids_{\lambda}, cid_{\lambda}),  \in \Lambda$
    \STATE $R^* = \{\forall r \in R \mid id_{r} \in rids_{\lambda}\}$
    \STATE $E^* = \{\forall e \in E \mid A_{e} \cap A_{R^*} \neq \emptyset\}$
    \STATE $m   =$ {\bf Parallel Composition($E^*$)}
    \STATE $g   =$ {\bf Modified Parallel Composition($m$, $R^*$)}
    \STATE $g^* =$ {\bf Error State Abstraction($g$)}
    \STATE $c   =$ {\bf Safety Game Solving($g^*$)}
    \STATE $c^* = (cid_{\lambda}, S_{c}, A_{c}, \Delta_{c}, s^0_{c})$
    \STATE $R   = R \backslash R^*$,\;\;\ $E = E \backslash E^* \cup \{c^*\}$
\ENDFOR
\STATE {\bf return} $c^* \in E$
\end{algorithmic}
\end{algorithm}

Algorithm\ref{algorithm:SDCS}では,$\Lambda$に則って,繰り返し部分合成(6-9行目)を適用することで$c$を導出する.
このとき,$\Lambda$は定義\ref{def:synthesis_sequence}で定義され,Algorithm\ref{algorithm:huristic}で導出される(1行目)\cite{yamauchi:KBSE2024}.

\begin{dfn}{\textbf{合成シーケンス }}
\label{def:synthesis_sequence}
    合成シーケンス$\Lambda$は合成プロセス$\lambda = (i_{\lambda}, eids_{\lambda}, rids_{\lambda}, cid_{\lambda})$の集合によって表される.
    $i_{\lambda}$は$\Lambda$において何番目に実行する$\lambda$であるかを表し,1から連続する整数が与えられる.
    $eids_{\lambda}$は$\lambda$で入力される環境モデル(LTS)の$id$の集合,$rids_{\lambda}$は$\lambda$で入力される監視モデル(LTS)の$id$の集合を表す.$cid_{\lambda}$は$\lambda$の出力となる部分制御器(LTS)$c$の$id$を表し,$\lambda$において$c$を合成した際には$id_{c}$にこの$cid_{\lambda}$を割り当てる.
\end{dfn}

\begin{algorithm}[h]
\caption{監視対象モデル数を考慮した$\Lambda$の合成}
\label{algorithm:huristic}
\begin{algorithmic}[1]
\renewcommand{\algorithmicrequire}{\textbf{Input:}}
\renewcommand{\algorithmicensure}{\textbf{Output:}}
\REQUIRE $E$ {\bf such that} $\forall e \in E, e = (id_{e}, S_{e}, A_{e}, \Delta_{e}, s^0_{e})$
\REQUIRE $R$ {\bf such that} $\forall r \in R, r = (id_{r}, S_{r}, A_{r}, \Delta_{r}, s^0_{r})$, $i^*$
\ENSURE  $\Lambda$
\STATE $\mu^*= \infty$,\;\; $cid^* = ${\bf uniqueID}
\FORALL{$r \in R$}
    \STATE $\mu = n(\{\forall e \in E \mid A_e \cap A_r \neq \emptyset\})$
    \IF{$\mu < \mu^*$}
        \STATE $r^* = r$,\;\; $\mu^* = \mu$
    \ENDIF
\ENDFOR
\STATE $E^* = \{\forall e \in E \mid A_{e} \cap A_{R^*} \neq \emptyset\}$
\STATE $\lambda^* = \{(i^*, \{\forall id_{e} \mid (id_{e}, S_{e}, A_{e}, \Delta_{e}, s^0_{e}) \in E^*\}, \{id_{r^*}\}, cid^*)\}$
\STATE $c^* = (cid^*, \emptyset, A_{E^*}, \emptyset, \emptyset)$
\IF{$n(R)=1$}
\STATE {\bf return} $\{\lambda^*\}$
\ENDIF
\STATE {\bf return} $\{\lambda^*\}$ $\cup$ {\bf Algorithm \ref{algorithm:huristic}($E \backslash E^* \cup \{c^*\}$, $R \backslash \{r^*\}$, $i^*+1$)}
\end{algorithmic}
\end{algorithm}

Algorithm\ref{algorithm:huristic}では,全監視モデルのうち$\mu$が最小となる監視モデル$r \in R$を$r^*$として導出し(2-7行目),全環境モデルのうち$r^*$の事象を含む全ての環境モデルを$E^*$として導出する(8行目).
そして,$r^*$と$E^*$から$i^*$ステップ目に部分合成する合成プロセスを構築する(9行目).
その後,$r^*$と$E^*$を部分合成した際に合成される制御器$c^*$を以降の合成プロセスの導出で必要な情報($id_{c^*}$,$A_{c^*}$)だけ埋めて仮に構築し(10行目),$i^*+1$ステップ目の合成プロセスの導出する(14行目).これを監視対象モデル数の数だけ繰り返し実施する(11-13行目).

以上のようにして導出された$\Lambda$に基づいて,Algorithm\ref{algorithm:huristic}では$i_{\lambda}$が1の$\lambda \in \Lambda$から順に繰り返し部分合成を行う(2-12行目).部分合成では,はじめに,入力された$R$のうち$rids_{\lambda}$に$id_{r}$が含まれる$r$の集合を$R^*$とし(4行目),$R^*$と$E$から部分合成の適用可能条件式$A_{r} \subseteq A_{E^*} \backslash A_{E \backslash E^*}$を満たす$e \in E$の集合を$E^*$として導出する(5行目).その後,導出された$R^*$と$E^*$を用いてParallel Composition(6行目),Modified Parallel Composition(7行目),Error State Abstraction(8行目),Safety Game Solving(9行目)を適用し$c$を合成する.そして,合成された$c$の$id$に$cid_{\lambda}$を割り当て$c^*$を用意する(10行目).最後に,次の$\lambda$で使用する$R$と$E$から使用した$R^*$と$E^*$を取り除き,$c^*$を$E$に加える(11行目).これにより,$i^*+1$ステップ目以降の部分合成において,$i^*$ステップ目の$R^*$を違反する状態空間の構築を回避することができる.

% ---------------------------------------------------------------------------------------------------------- %

\subsection{CDCSとSDCSにおける適用選択の課題}
\label{subsection:limitation}
CDCSやSDCSを用いて,basic DCSよりも効率よく制御器を合成するにあたり,開発者は以下の制約を考慮しつつ適切にCDCSやSDCSを使い分ける必要がある.

\begin{enumerate}[\bf 制約1]
\item CDCSまたはSDCSのいずれか一方のみが有効となる場合がある
\item[\bf 制約2]  SDCSはbasic DCSよりも計算時間が増加(性能が悪化)する可能性がある
\end{enumerate}

制約1は,CDCSとSDCSがそれぞれ異なる方針で計算空間の削減を行うため,削減可能な状態が異なり,適用可能範囲が一致しないことに起因する.
制約2は、SDCSにおいて計算空間の削減効果が限定的な場合,段階的なゲーム空間$g^{*}$の構築・分析を繰り返すオーバヘッドが相対的に大きくなり,全体の計算時間が悪化することに起因する.

以上の制約を踏まえて,CDCSとSDCSの適用選択は慎重に判断する必要がある.
しかし,判断には,各手法により生成される中間生成LTS,$m$,$g$,$g^{*}$の構造を詳細に把握する必要があり,これを手作業で行うことは実運用上の大きな障壁となっている.
本論文では,このSDCSとCDCSの適用選択の課題を解消することを目的とする.

% しかし,現状でこれらの制約を考慮するには,各手法で生成される複雑な中間生成LTS($m$,$g$,$g^{*}$)をシステムごとに手作業で把握するほかなく,CDCSやSDCSのどちらを適用するか判断する上で障壁となっている.
% 制約2は,SDCSにおいて,計算空間の削減効果が小さい場合,削減される計算時間よりも繰り返し$g^{*}$を構築・分析するオーバヘッドが優位となることに起因する.
% 上記制約を考慮してCDCSとSDCSの適用選択を慎重に判断する必要があるが,上記制約を考慮するためには各手法で生成される複雑な中間生成LTS($m$,$g$,$g^{*}$)を手作業で把握する他なく,実運用上の障壁となっている.

% 制約1は,CDCSとSDCSで削減できる計算空間が異なる(適用可能範囲が異なる)ことに起因する.

% CDCSは監視モデルにおける違反状態をひとつの状態に抽象化しつつ$g^{*}$を構築するため以下の特徴がある.
% \begin{enumerate}[\bf 削減対象:]
% \item $m$,$g$($g^{*}$構築過程で生成されるLTS)
% \item[\bf 限界:] $g^*$の構築過程のみを改善するため,basic DCSと比較して$g^{*}$の削減効果は得られない
% \end{enumerate}

% 一方,SDCSは監視モデルごとに繰り返し$g^{*}$を構築・分析するため以下の特徴がある.
% \begin{enumerate}[\bf 削減対象:]
% \item $m$,$g$,$g^{*}$(全ての中間生成LTS)
% \item[\bf 限界:] 繰り返し$g^{*}$を構築・分析するため,計算空間の削減効果が小さい場合はbasic DCSよりも多くの計算時間を要する
% \end{enumerate}

% 以上のように,SDCSとCDCSでは削減対象となる中間生成LTS($m$,$g$,$g^{*}$)が異なる.
% この削減対象の違いは,それぞれの手法が採用する削減アプローチの違いによるものであるが,どちらがより有効に機能するかは,扱う環境モデルおよび監視モデルの組み合わせに依存する.
% 特に,SDCSは繰り返し$g^{*}$を構築・分析する性質上,計算空間の削減効果が小さい場合には,かえってbasic DCSよりも多くの計算時間を要する限界も有する.



% そのため,開発者は対象システムごとにSDCSとCDCSのどちらがより有効かを適切に判断する必要がある.
% しかし,現状では各手法で生成される複雑な中間生成LTSをシステムごとに手作業で把握し,どちらを適用するか判断するほかなく,実運用上の障壁となっている.
% 本論文では,このようなSDCSとCDCSの適用選択に関する課題を解消することを目的とする.
