\section{あとがき}
\label{section:conclusion}
本論文では,CDCSとSDCSにおける適用選択の課題を解消するために,両手法と比較して同等以上の削減可能対象と状態空間削減効果を備えるCSDCSを提案した.
SDCSの段階的なゲーム空間構築アプローチを基盤としつつ,ゲーム空間構築処理にConsolidated Game Compositionを採用することで,従来トレードオフの関係にあった高い計算空間削減効果と短い計算時間の両立を実現した.
このCSDCSによって,SDCSとCDCSの使い分けに関する判断工程を不要とし,両者の適用選択における課題を解消した.
今後の研究課題としては,制御器の用途に応じてCSDCSを他の手法と組み合わせ,さらなる計算空間削減の実現に向けた拡張を検討していく.