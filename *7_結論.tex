\section{あとがき}
\label{section:conclusion}
本論文では,CDCSとSDCSにおける適用選択の課題を解消するために,両手法と比較して同等以上の適用可能範囲と状態空間削減効果を備えるCSDCSを提案した.
SDCSの段階的なゲーム空間構築アプローチを基盤としつつ,ゲーム空間構築処理にConsolidated Game Compositionを採用することで,高い計算空間削減効果と少ない計算時間の両立を実現した.
このCSDCSによって,SDCSとCDCSの使い分けに関する判断工程を不要とし,両者の適用選択における課題を解消した.
今後の研究課題としては,CSDCS アルゴリズムを制御器の用途に応じて最適化し,さらなる計算空間の削減を図ることが挙げられる.

% 本論文では,コンポーネントモデルと監視モデルから直接ゲーム空間を構築する直接的一括制御器合成を新たに提案し,ベーシックなDiscrete Controller Synthesis(DCS)の提要範囲と網羅性をそのままに計算空間や計算時間を削減した.本手法で新たに提案されたゲーム空間構築手法を用いることでモノリシックモデルをDCSにおいて構築する必要がなくなり,合成にかかる工数が簡略化され計算時間が短くなるだけでなく,モノリシックモデルがゲーム空間よりも状態空間が大きくなるDCSにおいて合成に必要となる計算空間が少なくなることがわかった.将来研究として,本研究の成果であるゲーム空間構築手法を関連研究であるStepwised Discrete Controller Synthesisにおいて適用し,更なる合成コストの削減を実現する離散制御器合成の定式化があげられる.
% \textbf{謝辞} 本研究はJSPS科研費(18H03225,17H00732)の助成を受けたものです。